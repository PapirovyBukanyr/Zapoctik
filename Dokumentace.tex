\documentclass{article}
\usepackage[utf8]{inputenc}
\usepackage[czech]{babel}
\usepackage{listings}
\usepackage{color}
\definecolor{keyword}{rgb}{0.0, 0.5, 0.0}
\lstset{basicstyle=\ttfamily,keywordstyle=\color{keyword},commentstyle=\itshape\color{blue},escapeinside={\%*}{*)}}
\begin{document}
\title{Dokumentace projektu}
\author{PříHoDa}
\date{\today}
\maketitle
\section*{Složka: .vscode}
\section*{Složka: C\_files}
\subsection*{Soubor: parse_python.py}
\begin{itemize}
 \subsection*{Třída: CommentExtractor}
\item \textit{Metoda: \\_\\_init\\_\\_}
\item \textit{Metoda: visit\\_FunctionDef}
\item \textit{Metoda: visit\\_ClassDef}
\end{itemize}
\section*{Složka: games}
\subsection*{Soubor: Board.py}
\begin{itemize}
 \subsection*{Třída: Board}
\begin{itemize}
\item{Třída reprezentující hrací desku hry
    }
\item{Konstruktor třídy Board
        }
\item{Inicializace herní desky
        }
\item{Vytvoří string hrací desky na výpis do konzole
        }
\item{Vrací šachovnici jako list

Returns:
    List of Struct : List, kde každý řádek je list obsahující figury na daném řádku}
\item{Pro možnost přstupovat k poli board jako board[row,col] místo board.board[row][col]

Args:
    index: Tuple dvou integerů, (row, col)
    
Returns:
    Figuru na určeném místě na šachovnici, případně None, pokud je prázdné}
\item{Nastaví políčko na šachovnici jako board[row,col] namísto board.board[row][col] 

Args:
    index ([int, int]): Tuple dvou integerů, (row, col)
    value (any): Co se má nastavit na dané políčko}
\end{itemize}
\item \textit{Metoda: \\_\\_init\\_\\_}
\textit{Konstruktor třídy Board
        }
\item \textit{Metoda: \\_\\_populateBoard}
\textit{Inicializace herní desky
        }
\item \textit{Metoda: \\_\\_str\\_\\_}
\textit{Vytvoří string hrací desky na výpis do konzole
        }
\item \textit{Metoda: getListOfBoard}
\textit{Vrací šachovnici jako list

Returns:
    List of Struct : List, kde každý řádek je list obsahující figury na daném řádku}
\item \textit{Metoda: \\_\\_getitem\\_\\_}
\textit{Pro možnost přstupovat k poli board jako board[row,col] místo board.board[row][col]

Args:
    index: Tuple dvou integerů, (row, col)
    
Returns:
    Figuru na určeném místě na šachovnici, případně None, pokud je prázdné}
\item \textit{Metoda: \\_\\_setitem\\_\\_}
\textit{Nastaví políčko na šachovnici jako board[row,col] namísto board.board[row][col] 

Args:
    index ([int, int]): Tuple dvou integerů, (row, col)
    value (any): Co se má nastavit na dané políčko}
\end{itemize}
\subsection*{Soubor: ChallengeAccepted.py}
\begin{itemize}
 \subsection*{Třída: ChallengeAccepted}
\item \textit{Atribut: \\_\\_init\\_\\_}
\item \textit{Atribut: getBoard}
\item \textit{Atribut: getField}
\item \textit{Atribut: moveMole}
\item \textit{Atribut: makeMove}
\item \textit{Atribut: checkEnd}
\item \textit{Atribut: printToTerminal}
\end{itemize}
\section*{Složka: checkers}
\subsection*{Soubor: Checkers.py}
\begin{itemize}
 \subsection*{Třída: Checkers}
\begin{itemize}
\item{Třída reprezentující hru dáma

Attributes:
    withChoosePiece (bool): True, pokud je v hře možné vybrat figurku, jinak False
    numberOfPlayers (int): počet hráčů
    fog (bool): True, pokud je ve hře Fog Of War, jinak False
    \\_\\_board (CheckersBoard): hrací deska
    \\_\\_currentPlayer (Enum Colors): barva hráče na tahu
    \\_\\_pieceToPlay (Piece): figurka, kterou hráč chce hrát
    \\_\\_firstMove (bool): True, pokud je první tah, jinak False}
\item{Konstruktor třídy hry dáma}
\item{Funkce pro získání hrací desky

Args:
    color (Enum Colors, optional): barva hráče na tahu. Výchozí nastavení je na pravidelném střídání.
    
Returns:
    List of List of [int, int]: hrací deska}
\item{Funkce pro vyber figurky, kterou chce hrac hrat

Args:
    index ([int,int]): pozice figurky, kterou chce hrac hrat
    color (Enum Colors, optional): barva hráče na tahu. Výchozí nastavení je na pravidelném střídání.

Returns:
    List of [int, int]: seznam dostupných pozic, kam může hráč hrát}
\item{Funkce pro provedení tahu figurkou

Args:
    index ([int,int]): pozice, kam chce hráč hrát
    color (Enum Colors, optional): barva hráče na tahu. Výchozí nastavení je na pravidelném střídání.
    rightClick (bool, optional): True, pokud se jedná o pravé tlačítko myši. Výchozí nastavení je False.
    
Returns:
    bool: True, pokud se tah podařil, jinak False}
\item{Funkce pro resetování hry}
\item{Funkce pro kontrolu konce hry

Returns:
    string: Vrací vítěze "{barva} won", pokud hra skončila, jinak None}
\item{Funkce pro ukončení tahu}
\item{Funkce pro získání možných tahů pro hráče, primárně pro rozšíření Fog Of War

Args:
    color (Enum Colors): barva hráče, pro kterého se mají tahy získat
    
Returns:
    List of [int, int]: seznam možných tahů}
\item{Funkce pro odstranění figurky z hrací desky

Args:
    piecePosition ([int, int]): pozice figurky, která má být odstraněna}
\item{Funkce pro výpis stavu hry na terminál}
\end{itemize}
\item \textit{Metoda: \\_\\_init\\_\\_}
\textit{Konstruktor třídy hry dáma}
\item \textit{Metoda: getBoard}
\textit{Funkce pro získání hrací desky

Args:
    color (Enum Colors, optional): barva hráče na tahu. Výchozí nastavení je na pravidelném střídání.
    
Returns:
    List of List of [int, int]: hrací deska}
\item \textit{Metoda: choosePiece}
\textit{Funkce pro vyber figurky, kterou chce hrac hrat

Args:
    index ([int,int]): pozice figurky, kterou chce hrac hrat
    color (Enum Colors, optional): barva hráče na tahu. Výchozí nastavení je na pravidelném střídání.

Returns:
    List of [int, int]: seznam dostupných pozic, kam může hráč hrát}
\item \textit{Metoda: makeMove}
\textit{Funkce pro provedení tahu figurkou

Args:
    index ([int,int]): pozice, kam chce hráč hrát
    color (Enum Colors, optional): barva hráče na tahu. Výchozí nastavení je na pravidelném střídání.
    rightClick (bool, optional): True, pokud se jedná o pravé tlačítko myši. Výchozí nastavení je False.
    
Returns:
    bool: True, pokud se tah podařil, jinak False}
\item \textit{Metoda: reset}
\textit{Funkce pro resetování hry}
\item \textit{Metoda: checkEnd}
\textit{Funkce pro kontrolu konce hry

Returns:
    string: Vrací vítěze "{barva} won", pokud hra skončila, jinak None}
\item \textit{Metoda: \\_\\_endOfTurn}
\textit{Funkce pro ukončení tahu}
\item \textit{Metoda: possibleMoves}
\textit{Funkce pro získání možných tahů pro hráče, primárně pro rozšíření Fog Of War

Args:
    color (Enum Colors): barva hráče, pro kterého se mají tahy získat
    
Returns:
    List of [int, int]: seznam možných tahů}
\item \textit{Metoda: killPiece}
\textit{Funkce pro odstranění figurky z hrací desky

Args:
    piecePosition ([int, int]): pozice figurky, která má být odstraněna}
\item \textit{Metoda: \\_\\_printToTerminal}
\textit{Funkce pro výpis stavu hry na terminál}
\end{itemize}
\subsection*{Soubor: CheckersBoard.py}
\begin{itemize}
 \subsection*{Třída: CheckersBoard}
\begin{itemize}
\item{Třída reprezentující šachovnici pro hru dáma. Dědí z třídy Board.

Attributes:
    board (List of List of Pieces): hrací deska, kde None reprezentuje prázdné políčko}
\item{Konstruktor třídy CheckersBoard. Vytvoří šachovnici a umístí na ni všechny figurky do počáteční polohy.}
\item{Vrací figuru na určeném místě na šachovnici, nebo None, pokud je políčko prázdné. 

Args:
    index [int, int]: Tuple dvou integerů, (row, col), oba 0-7}
\item{Nastaví políčko na šachovnici jako board[row,col] namísto board.board[row][col] 

Args:
    index ([int, int]): Tuple dvou integerů, (row, col), oba 0-7
    value (Piece): Instance třídy Piece, nebo None, pokud má být políčko prázdné
    
Returns:
    bool: True, pokud se podařilo nastavit políčko, jinak False}
\item{    Vrací string reprezentaci šachovnice. Každé políčko je reprezentováno jako string, který je tvořen z informací o barvě a symbolu figury, nebo jako string "\\_\\_", pokud je políčko prázdné. Políčka jsou oddělena mezerou a jednotlivé řádky jsou odděleny znakem nového řádku (
).
    
    Returns:
        string: string reprezentace šachovnice
    }
\item{Nastaví šachovnici do počátečního stavu. Bílý je dole a černý nahoře.}
\item{Vrací list všech figurek dané barvy na šachovnici.

Args:
    color (Enum Colors): Barva figurek, které chceme najít (Colors.WHITE nebo Colors.BLACK)
    
Returns: 
    List of Pieces: List figurek dané barvy}
\item{Vrací list všech figurek na šachovnici.

Returns: 
    List of Fields: List figurek na šachovnici}
\end{itemize}
\item \textit{Metoda: \\_\\_init\\_\\_}
\textit{Konstruktor třídy CheckersBoard. Vytvoří šachovnici a umístí na ni všechny figurky do počáteční polohy.}
\item \textit{Metoda: \\_\\_getitem\\_\\_}
\textit{Vrací figuru na určeném místě na šachovnici, nebo None, pokud je políčko prázdné. 

Args:
    index [int, int]: Tuple dvou integerů, (row, col), oba 0-7}
\item \textit{Metoda: \\_\\_setitem\\_\\_}
\textit{Nastaví políčko na šachovnici jako board[row,col] namísto board.board[row][col] 

Args:
    index ([int, int]): Tuple dvou integerů, (row, col), oba 0-7
    value (Piece): Instance třídy Piece, nebo None, pokud má být políčko prázdné
    
Returns:
    bool: True, pokud se podařilo nastavit políčko, jinak False}
\item \textit{Metoda: \\_\\_str\\_\\_}
\textit{    Vrací string reprezentaci šachovnice. Každé políčko je reprezentováno jako string, který je tvořen z informací o barvě a symbolu figury, nebo jako string "\\_\\_", pokud je políčko prázdné. Políčka jsou oddělena mezerou a jednotlivé řádky jsou odděleny znakem nového řádku (
).
    
    Returns:
        string: string reprezentace šachovnice
    }
\item \textit{Metoda: \\_\\_populateBoard}
\textit{Nastaví šachovnici do počátečního stavu. Bílý je dole a černý nahoře.}
\item \textit{Metoda: pieceList}
\textit{Vrací list všech figurek dané barvy na šachovnici.

Args:
    color (Enum Colors): Barva figurek, které chceme najít (Colors.WHITE nebo Colors.BLACK)
    
Returns: 
    List of Pieces: List figurek dané barvy}
\item \textit{Metoda: getListOfBoard}
\textit{Vrací list všech figurek na šachovnici.

Returns: 
    List of Fields: List figurek na šachovnici}
\end{itemize}
\subsection*{Soubor: CheckersMines.py}
\begin{itemize}
 \subsection*{Třída: CheckersMines}
\begin{itemize}
\item{Hra dáma s minami.

Attributes:
    mines (List of [int, int]): seznam pozic min
    explosion ([int, int]): pozice exploze
    withChoosePiece (bool): True, pokud je v hře možné vybrat figurku, jinak False
    numberOfPlayers (int): počet hráčů
    fog (bool): True, pokud je ve hře Fog Of War, jinak False}
\item{Inicializace hry.}
\item{Umístí miny na náhodné pozice.}
\item{Vrátí hrací desku.}
\item{Zpracuje tah.}
\end{itemize}
\item \textit{Metoda: \\_\\_init\\_\\_}
\textit{Inicializace hry.}
\item \textit{Metoda: \\_\\_placeMines}
\textit{Umístí miny na náhodné pozice.}
\item \textit{Metoda: getBoard}
\textit{Vrátí hrací desku.}
\item \textit{Metoda: makeMove}
\textit{Zpracuje tah.}
\end{itemize}
\subsection*{Soubor: CheckersMinesWithFogOfWar.py}
\begin{itemize}
 \subsection*{Třída: CheckersMinesWithFogOfWar}
\begin{itemize}
\item{Třída CheckersWithFogOfWar slouží k reprezentaci hry Dáma s mlhou války.

Attributes:
    mines (List of [int, int]): seznam pozic min
    explosion ([int, int]): pozice exploze
    withChoosePiece (bool): True, pokud je v hře možné vybrat figurku, jinak False
    numberOfPlayers (int): počet hráčů
    fog (bool): True, pokud je ve hře Fog Of War, jinak False}
\item{Konstruktor třídy CheckersWithFogOfWar}
\item{Vrací zakrytou šachovnici

Args:
    color (Enum Colors): Barva hráče na tahu

Returns:
    Array of Field: zakrytá šachovnice}
\end{itemize}
\item \textit{Metoda: \\_\\_init\\_\\_}
\textit{Konstruktor třídy CheckersWithFogOfWar}
\item \textit{Metoda: getBoard}
\textit{Vrací zakrytou šachovnici

Args:
    color (Enum Colors): Barva hráče na tahu

Returns:
    Array of Field: zakrytá šachovnice}
\end{itemize}
\section*{Složka: pieces}
\subsection*{Soubor: Pawn.py}
\begin{itemize}
 \subsection*{Třída: Pawn}
\begin{itemize}
\item{Třída reprezentující pěšáka

Attributes:
    color (Colors): barva figurky
    position ([int,int]): pozice figurky}
\item{Funkce na výpis figurky

Returns:
    str: P + barva figurky}
\item{Funkce najde všechny možné tahy figurky

Args:
    board (two-dimensional array of ints): hrací deska
    
Returns:
    [int,int]: pole možných tahů}
\item{Funkce najde všechny jednoduché skoky z pozice

Args:
    board (two-dimensional array of ints): hrací deska
    position ([int,int]], optional): \\_description\\_. Defaults to None.

Returns:
    [int,int]: pole možných skoků}
\item{Funkce na zjištění přeskočených figurek

Args:
    endPosition ([int, int]): koncová pozice

Returns:
    [int, int]: pozice přeskoečené figurky}
\end{itemize}
\item \textit{Metoda: \\_\\_str\\_\\_}
\textit{Funkce na výpis figurky

Returns:
    str: P + barva figurky}
\item \textit{Metoda: possibleMoves}
\textit{Funkce najde všechny možné tahy figurky

Args:
    board (two-dimensional array of ints): hrací deska
    
Returns:
    [int,int]: pole možných tahů}
\item \textit{Metoda: possibleJumps}
\textit{Funkce najde všechny jednoduché skoky z pozice

Args:
    board (two-dimensional array of ints): hrací deska
    position ([int,int]], optional): \\_description\\_. Defaults to None.

Returns:
    [int,int]: pole možných skoků}
\item \textit{Metoda: trackJumps}
\textit{Funkce na zjištění přeskočených figurek

Args:
    endPosition ([int, int]): koncová pozice

Returns:
    [int, int]: pozice přeskoečené figurky}
\end{itemize}
\subsection*{Soubor: Piece.py}
\begin{itemize}
 \subsection*{Třída: Piece}
\begin{itemize}
\item{Třída reprezentující figurku na šachovnici

Attributes:
    color (Colors): barva figurky
    position ([int,int]): pozice figurky}
\item{Konstruktor třídy Piece

Args:
    color (Enum Colors): požadovaná barva figurky
    position ([int,int]): výchozí pozice figurky}
\item{Metoda pro zjištění možných tahů figurky

Args:
    board (Board): hrací deska, na které se figurka nachází}
\item{Metoda pro zjištění možných skoků figurky

Args:
    board (Board): hrací deska, na které se figurka nachází
    position ([int, int], optional): pozice, ze které se má skákat. Výchozí hodnota je None, což znamená, že se skáče ze současné pozice figurky}
\item{Metoda pro zjištění možných skoků figurky

Args:
    board (Board): hrací deska, na které se figurka nachází
    endPosition ([int, int]): pozice, kam se má skákat}
\item{Vlastnost pro získání řádku, na kterém se figurka nachází

Returns:
    int: řádek, na kterém se figurka nachází}
\item{Vlastnost pro získání sloupce, na kterém se figurka nachází

Returns:
    int: sloupec, na kterém se figurka nachází}
\end{itemize}
\item \textit{Metoda: \\_\\_init\\_\\_}
\textit{Konstruktor třídy Piece

Args:
    color (Enum Colors): požadovaná barva figurky
    position ([int,int]): výchozí pozice figurky}
\item \textit{Metoda: possibleMoves}
\textit{Metoda pro zjištění možných tahů figurky

Args:
    board (Board): hrací deska, na které se figurka nachází}
\item \textit{Metoda: possibleJumps}
\textit{Metoda pro zjištění možných skoků figurky

Args:
    board (Board): hrací deska, na které se figurka nachází
    position ([int, int], optional): pozice, ze které se má skákat. Výchozí hodnota je None, což znamená, že se skáče ze současné pozice figurky}
\item \textit{Metoda: trackJumps}
\textit{Metoda pro zjištění možných skoků figurky

Args:
    board (Board): hrací deska, na které se figurka nachází
    endPosition ([int, int]): pozice, kam se má skákat}
\item \textit{Metoda: row}
\textit{Vlastnost pro získání řádku, na kterém se figurka nachází

Returns:
    int: řádek, na kterém se figurka nachází}
\item \textit{Metoda: col}
\textit{Vlastnost pro získání sloupce, na kterém se figurka nachází

Returns:
    int: sloupec, na kterém se figurka nachází}
\end{itemize}
\subsection*{Soubor: Queen.py}
\begin{itemize}
 \subsection*{Třída: Queen}
\begin{itemize}
\item{Třída reprezentující dámu

Attributes:
    color (Colors): barva figurky
    position ([int,int]): pozice figurky}
\item{Konstruktor třídy Queen

Args:
    pawn (Pawn): Pěšák, který se má proměnit na dámu}
\item{Vrací string reprezentaci dámy

Returns:
    string: string reprezentace dámy}
\item{Vrací seznam možných tahů dámy

Args:
    board (CheckersBoard): šachovnice, na které se dáma nachází

Returns:
    List of [int, int] : seznam možných tahů dámy}
\item{Vrací seznam možných prvních skoků dámy

Args:
    board (CheckersBoard): šachovnice, na které se dáma nachází
    position ([int, int], optional): pozice, ze které se má skákat. Výchozí hodnota je None, což znamená, že se skáče ze současné pozice dámy
    
Returns:
    List of [int, int]: seznam možných prvních skoků dámy}
\item{Vrací seznam pozic figurek, které dáma přeskočí, než se dostane na koncovou pozici

Args:
    endPosition ([int, int]): koncová pozice, na kterou se má dáma dostat
    
Returns:
    List of [int, int]: seznam pozic figurek, které dáma přeskočí, než se dostane na koncovou pozici}
\end{itemize}
\item \textit{Metoda: \\_\\_init\\_\\_}
\textit{Konstruktor třídy Queen

Args:
    pawn (Pawn): Pěšák, který se má proměnit na dámu}
\item \textit{Metoda: \\_\\_str\\_\\_}
\textit{Vrací string reprezentaci dámy

Returns:
    string: string reprezentace dámy}
\item \textit{Metoda: possibleMoves}
\textit{Vrací seznam možných tahů dámy

Args:
    board (CheckersBoard): šachovnice, na které se dáma nachází

Returns:
    List of [int, int] : seznam možných tahů dámy}
\item \textit{Metoda: possibleJumps}
\textit{Vrací seznam možných prvních skoků dámy

Args:
    board (CheckersBoard): šachovnice, na které se dáma nachází
    position ([int, int], optional): pozice, ze které se má skákat. Výchozí hodnota je None, což znamená, že se skáče ze současné pozice dámy
    
Returns:
    List of [int, int]: seznam možných prvních skoků dámy}
\item \textit{Metoda: trackJumps}
\textit{Vrací seznam pozic figurek, které dáma přeskočí, než se dostane na koncovou pozici

Args:
    endPosition ([int, int]): koncová pozice, na kterou se má dáma dostat
    
Returns:
    List of [int, int]: seznam pozic figurek, které dáma přeskočí, než se dostane na koncovou pozici}
\end{itemize}
\subsection*{Soubor: __init__.py}
\begin{itemize}
\end{itemize}
\section*{Složka: \_\_pycache\_\_}
\subsection*{Soubor: __init__.py}
\begin{itemize}
\end{itemize}
\section*{Složka: \_\_pycache\_\_}
\subsection*{Soubor: CheckersWithFogOfWar.py}
\begin{itemize}
 \subsection*{Třída: CheckersWithFogOfWar}
\begin{itemize}
\item{Třída CheckersWithFogOfWar slouží k reprezentaci hry Dáma s mlhou války.
    }
\item{Konstruktor třídy CheckersWithFogOfWar
        }
\item{Vrací zakrytou šachovnici

Args:
    color (Enum Colors): Barva hráče na tahu

Returns:
    Array of Field: zakrytá šachovnice}
\end{itemize}
\item \textit{Metoda: \\_\\_init\\_\\_}
\textit{Konstruktor třídy CheckersWithFogOfWar
        }
\item \textit{Metoda: getBoard}
\textit{Vrací zakrytou šachovnici

Args:
    color (Enum Colors): Barva hráče na tahu

Returns:
    Array of Field: zakrytá šachovnice}
\end{itemize}
\section*{Složka: chess}
\subsection*{Soubor: Chess.py}
\begin{itemize}
 \subsection*{Třída: Chess}
\begin{itemize}
\item{Třída reprezentující hru šachy

Attributes:
  withChoosePiece (bool): True, pokud se má hráči zobrazovat možnost vybrat figurku, jinak False
  \\_\\_movesSinceLastImportantMove (int): Počet tahů od posledního důležitého tahu
  \\_\\_board (ChessBoard): Šachovnice
  \\_\\_playedPiece (Piece): Figurka, kterou hráč právě hraje
  \\_\\_positionsList (list of ChessBoard): Seznam všech pozic na šachovnici
  \\_\\_isMoving (Colors): Barva hráče, který je na tahu}
\item{Konstruktor třídy šachů}
\item{Vrátí šachovnici v aktuálním stavu jako dvourozměrné pole Field

Args:
    color (Enum Colors): Barva hráče, pro kterého se má šachovnice vykreslit
    
Returns:
    list of list of Field: šachovnice}
\item{Funkce pro vyber figurky, kterou chce hrac hrat

Args:
    positionToPlay ([int, int]): pozice figurky, kterou chce hrac hrat
    color (Enum Colors): Barva hrace, ktery chce hrat

Returns:
    (list of [int, int]): dostupne pozice, kam muze hrac hrat}
\item{Provedení tahu hrace

Args:
    playedMove ([int, int]): pozice, kam chce hrac hrat
    color (Enum Colors): Barva hrace, ktery chce hrat
    rightClick (bool): True, pokud hrac klikl pravym tlacitkem mysi, jinak False

Returns:
    bool: tah se zdařil nebo ne
    string: "Promote" pokud je potreba provest vylepseni pesaka}
\item{Promote pesaka

Args:
    newFigure (string): figurka, na kterou se ma pesak zmenit ("Q", "R", "B", "N")
    
Returns:
    bool: podle toho zda se tah podaril nebo ne
    string: pokud hra skoncila}
\item{Konec tahu

Returns:
    True: novy stav sachovnice po tahu
    string: string pokud hra skoncila}
\item{Vrati mozne tahy pro hrace, primárně pro rozšíření Fog Of War

Args:
    color (Enum Colors): barva hrace, pro ktereho se maji tahy vypsat
    
Returns:
    list of [int, int]: seznam moznych tahu}
\item{Kontrola konce hry

Returns:
    string: "Draw by fifty-move rule" pokud bylo 50 tahu bez pohybu pesaku nebo braneni
    string: "Draw by threefold repetition" pokud se stejna pozice opakovala 3x
    string: "{color} won" pokud byl vyhozen kral
    None: pokud hra neskoncila}
\item{Vyhození figurky z hrací desky

Args:
    piecePosition ([int, int]): pozice figurky, kterou chceme vyhodit}
\item{Vytiskne hrací desku do konzole}
\end{itemize}
\item \textit{Metoda: \\_\\_init\\_\\_}
\textit{Konstruktor třídy šachů}
\item \textit{Metoda: getBoard}
\textit{Vrátí šachovnici v aktuálním stavu jako dvourozměrné pole Field

Args:
    color (Enum Colors): Barva hráče, pro kterého se má šachovnice vykreslit
    
Returns:
    list of list of Field: šachovnice}
\item \textit{Metoda: choosePiece}
\textit{Funkce pro vyber figurky, kterou chce hrac hrat

Args:
    positionToPlay ([int, int]): pozice figurky, kterou chce hrac hrat
    color (Enum Colors): Barva hrace, ktery chce hrat

Returns:
    (list of [int, int]): dostupne pozice, kam muze hrac hrat}
\item \textit{Metoda: makeMove}
\textit{Provedení tahu hrace

Args:
    playedMove ([int, int]): pozice, kam chce hrac hrat
    color (Enum Colors): Barva hrace, ktery chce hrat
    rightClick (bool): True, pokud hrac klikl pravym tlacitkem mysi, jinak False

Returns:
    bool: tah se zdařil nebo ne
    string: "Promote" pokud je potreba provest vylepseni pesaka}
\item \textit{Metoda: promote}
\textit{Promote pesaka

Args:
    newFigure (string): figurka, na kterou se ma pesak zmenit ("Q", "R", "B", "N")
    
Returns:
    bool: podle toho zda se tah podaril nebo ne
    string: pokud hra skoncila}
\item \textit{Metoda: \\_\\_endOfMove}
\textit{Konec tahu

Returns:
    True: novy stav sachovnice po tahu
    string: string pokud hra skoncila}
\item \textit{Metoda: possibleMoves}
\textit{Vrati mozne tahy pro hrace, primárně pro rozšíření Fog Of War

Args:
    color (Enum Colors): barva hrace, pro ktereho se maji tahy vypsat
    
Returns:
    list of [int, int]: seznam moznych tahu}
\item \textit{Metoda: checkEnd}
\textit{Kontrola konce hry

Returns:
    string: "Draw by fifty-move rule" pokud bylo 50 tahu bez pohybu pesaku nebo braneni
    string: "Draw by threefold repetition" pokud se stejna pozice opakovala 3x
    string: "{color} won" pokud byl vyhozen kral
    None: pokud hra neskoncila}
\item \textit{Metoda: killPiece}
\textit{Vyhození figurky z hrací desky

Args:
    piecePosition ([int, int]): pozice figurky, kterou chceme vyhodit}
\item \textit{Metoda: \\_\\_printToTerminal}
\textit{Vytiskne hrací desku do konzole}
\end{itemize}
\subsection*{Soubor: ChessBoard.py}
\begin{itemize}
 \subsection*{Třída: ChessBoard}
\begin{itemize}
\item{Třída reprezentující šachovnici s figurkami. Dědí od třídy Board. 

Attributes:
  board (list of list of Piece): Dvourozměrné pole, které reprezentuje šachovnici. Každé políčko obsahuje instanci třídy Piece, nebo None, pokud je políčko prázdné.}
\item{Konstruktor třídy ChessBoard. Vytvoří šachovnici 8x8 a umístí na ni všechny figury ve standardním pořadí.}
\item{Vrací figuru na určeném místě na šachovnici, nebo None, pokud je políčko prázdné. 

Args:
    index [int, int]: Tuple dvou integerů, (row, col), oba 0-7

Returns:
    Piece: Instance třídy Piece, nebo None, pokud je políčko prázdné}
\item{Nastaví políčko na šachovnici jako board[row,col] namísto board.board[row][col] 

Args:
    index ([int, int]): Tuple dvou integerů, (row, col), oba 0-7
    value (Piece): Instance třídy Piece, nebo None, pokud má být políčko prázdné
    
Returns:
    bool: True, pokud se podařilo nastavit políčko, jinak False}
\item{    Vrací string reprezentaci šachovnice. Každé políčko je reprezentováno jako string, který je tvořen z informací o barvě a symbolu figury, nebo jako string "\\_\\_", pokud je políčko prázdné. Políčka jsou oddělena mezerou a jednotlivé řádky jsou odděleny znakem nového řádku (
).
    }
\item{Nastaví šachovnici do normálního stavu. Všichni pěšáci jsou v druhém a sedmém řádku, všechny ostatní figury jsou v prvním a osmém řádku. Barva figurek je v souladu s konvencí, že bílý je dole a černý nahoře.}
\item{Vrací list všech figurek dané barvy na šachovnici.

Args:
    color (Enum Colors): Barva figurek, které chceme najít (Colors.WHITE nebo Colors.BLACK)
    
Returns: 
    List of Pieces: List figurek dané barvy}
\item{Vrátí kopii šachovnice. Každá figura z originální šachovnice je nahrazena její kopií.

Returns:
    ChessBoard: Kopie šachovnice}
\item{Porovná dvě šachovnice.
Tato metoda porovná dvě šachovnice a vrátí True, pokud jsou stejné. Jinak vrátí False.

Args: 
    board: Šachovnice, která se má porovnat s aktuální šachovnicí

Returns: 
    Boolean:True, pokud jsou šachovnice stejné, jinak False}
\item{Vrátí šachovnici jako list of Field.

Returns:
    List of Struct Field: šachovnice jako list of Field}
\end{itemize}
\item \textit{Metoda: \\_\\_init\\_\\_}
\textit{Konstruktor třídy ChessBoard. Vytvoří šachovnici 8x8 a umístí na ni všechny figury ve standardním pořadí.}
\item \textit{Metoda: \\_\\_getitem\\_\\_}
\textit{Vrací figuru na určeném místě na šachovnici, nebo None, pokud je políčko prázdné. 

Args:
    index [int, int]: Tuple dvou integerů, (row, col), oba 0-7

Returns:
    Piece: Instance třídy Piece, nebo None, pokud je políčko prázdné}
\item \textit{Metoda: \\_\\_setitem\\_\\_}
\textit{Nastaví políčko na šachovnici jako board[row,col] namísto board.board[row][col] 

Args:
    index ([int, int]): Tuple dvou integerů, (row, col), oba 0-7
    value (Piece): Instance třídy Piece, nebo None, pokud má být políčko prázdné
    
Returns:
    bool: True, pokud se podařilo nastavit políčko, jinak False}
\item \textit{Metoda: \\_\\_str\\_\\_}
\textit{    Vrací string reprezentaci šachovnice. Každé políčko je reprezentováno jako string, který je tvořen z informací o barvě a symbolu figury, nebo jako string "\\_\\_", pokud je políčko prázdné. Políčka jsou oddělena mezerou a jednotlivé řádky jsou odděleny znakem nového řádku (
).
    }
\item \textit{Metoda: \\_\\_populateBoard}
\textit{Nastaví šachovnici do normálního stavu. Všichni pěšáci jsou v druhém a sedmém řádku, všechny ostatní figury jsou v prvním a osmém řádku. Barva figurek je v souladu s konvencí, že bílý je dole a černý nahoře.}
\item \textit{Metoda: pieceList}
\textit{Vrací list všech figurek dané barvy na šachovnici.

Args:
    color (Enum Colors): Barva figurek, které chceme najít (Colors.WHITE nebo Colors.BLACK)
    
Returns: 
    List of Pieces: List figurek dané barvy}
\item \textit{Metoda: copy}
\textit{Vrátí kopii šachovnice. Každá figura z originální šachovnice je nahrazena její kopií.

Returns:
    ChessBoard: Kopie šachovnice}
\item \textit{Metoda: compare}
\textit{Porovná dvě šachovnice.
Tato metoda porovná dvě šachovnice a vrátí True, pokud jsou stejné. Jinak vrátí False.

Args: 
    board: Šachovnice, která se má porovnat s aktuální šachovnicí

Returns: 
    Boolean:True, pokud jsou šachovnice stejné, jinak False}
\item \textit{Metoda: getListOfBoard}
\textit{Vrátí šachovnici jako list of Field.

Returns:
    List of Struct Field: šachovnice jako list of Field}
\end{itemize}
\subsection*{Soubor: ChessMines.py}
\begin{itemize}
 \subsection*{Třída: ChessMines}
\begin{itemize}
\item{Hra šachy s minami.

Attributes:
    mines (list of [int, int]): seznam min
    explosion ([int, int]): pozice exploze}
\item{Inicializace hry.}
\item{Umístí miny na náhodné pozice.}
\item{Vrátí hrací desku.}
\item{Zpracuje tah.}
\end{itemize}
\item \textit{Metoda: \\_\\_init\\_\\_}
\textit{Inicializace hry.}
\item \textit{Metoda: \\_\\_placeMines}
\textit{Umístí miny na náhodné pozice.}
\item \textit{Metoda: getBoard}
\textit{Vrátí hrací desku.}
\item \textit{Metoda: makeMove}
\textit{Zpracuje tah.}
\end{itemize}
\subsection*{Soubor: ChessMinesWithFogOfWar.py}
\begin{itemize}
 \subsection*{Třída: ChessMinesWithFogOfWar}
\begin{itemize}
\item{Hra šachy s minami a mlhou války.

Attributes:
    fog (bool): True, pokud je mlha války zapnutá, jinak False}
\item{Konstruktor třídy ChessMinesWithFogOfWar}
\item{Vrátí hrací desku.

Args:
    color (Enum Colors): Barva hráče, pro kterého se má šachovnice vykreslit}
\end{itemize}
\item \textit{Metoda: \\_\\_init\\_\\_}
\textit{Konstruktor třídy ChessMinesWithFogOfWar}
\item \textit{Metoda: getBoard}
\textit{Vrátí hrací desku.

Args:
    color (Enum Colors): Barva hráče, pro kterého se má šachovnice vykreslit}
\end{itemize}
\section*{Složka: pieces}
\subsection*{Soubor: Bishop.py}
\begin{itemize}
 \subsection*{Třída: Bishop}
\begin{itemize}
\item{Třída reprezentující figurku střelce v šachu. Dědí od třídy Piece.

Attributes:
  color (Colors): barva figurky
  position ([int,int]): pozice figurky
  symbol (str): symbol figurky
  value (int): hodnota figurky}
\item{Konstruktor třídy Bishop. Volá konstruktor třídy Piece.

Args:
  color (Colors): barva figurky
  position ([int,int]): pozice figurky}
\item{Vytvoří kopii instance třídy Bishop.

Returns:
  Bishop: kopie instance třídy Bishop}
\item{Vrátí seznam možných tahů pro střelce. 

Args:
  board (Board): šachovnice, na které se figurka nachází}
\end{itemize}
\item \textit{Metoda: \\_\\_init\\_\\_}
\textit{Konstruktor třídy Bishop. Volá konstruktor třídy Piece.

Args:
  color (Colors): barva figurky
  position ([int,int]): pozice figurky}
\item \textit{Metoda: copy}
\textit{Vytvoří kopii instance třídy Bishop.

Returns:
  Bishop: kopie instance třídy Bishop}
\item \textit{Metoda: possibleMoves}
\textit{Vrátí seznam možných tahů pro střelce. 

Args:
  board (Board): šachovnice, na které se figurka nachází}
\end{itemize}
\subsection*{Soubor: King.py}
\begin{itemize}
 \subsection*{Třída: King}
\begin{itemize}
\item{Třída reprezentující figurku krále v šachu. Dědí od třídy Piece.

Attributes:
  color (Colors): barva figurky
  position ([int,int]): pozice figurky
  symbol (str): symbol figurky
  hasMoved (bool): True, pokud figurka už byla pohnuta, jinak False}
\item{Konstruktor třídy King. Volá konstruktor třídy Piece.

Args:
  color (Colors): barva figurky
  position ([int,int]): pozice figurky}
\item{Vytvoří kopii instance třídy King.

Returns:
  King: kopie instance třídy King}
\item{Zkontroluje, zda je možné provést tah králem a provede ho.

Args:
  board - šachovnice
  end - cílová pozice tahu}
\item{Vrátí seznam možných tahů pro krále. 

Args:
  board (Board): šachovnice, na které se figurka nachází
  
Returns:
  List of [int, int] : seznam možných tahů krále}
\end{itemize}
\item \textit{Metoda: \\_\\_init\\_\\_}
\textit{Konstruktor třídy King. Volá konstruktor třídy Piece.

Args:
  color (Colors): barva figurky
  position ([int,int]): pozice figurky}
\item \textit{Metoda: copy}
\textit{Vytvoří kopii instance třídy King.

Returns:
  King: kopie instance třídy King}
\item \textit{Metoda: move}
\textit{Zkontroluje, zda je možné provést tah králem a provede ho.

Args:
  board - šachovnice
  end - cílová pozice tahu}
\item \textit{Metoda: possibleMoves}
\textit{Vrátí seznam možných tahů pro krále. 

Args:
  board (Board): šachovnice, na které se figurka nachází
  
Returns:
  List of [int, int] : seznam možných tahů krále}
\end{itemize}
\subsection*{Soubor: Knight.py}
\begin{itemize}
 \subsection*{Třída: Knight}
\begin{itemize}
\item{Třída reprezentující figurku jezdce v šachu. Dědí od třídy Piece.

Attributes:
  color (Colors): barva figurky
  position ([int,int]): pozice figurky
  symbol (str): symbol figurky
  value (int): hodnota figurky}
\item{Konstruktor třídy Knight. Volá konstruktor třídy Piece

Args:
  color (Colors): barva figurky
  position ([int, int]): pozice figurky}
\item{Vytvoří kopii instance třídy Knight.

Returns:
  Knight: kopie instance třídy Knight}
\item{Vrátí seznam možných tahů pro jezdce.

Args:
  board (dict): šachovnice
  
Returns:
  List of [int,int]: seznam možných tahů  }
\end{itemize}
\item \textit{Metoda: \\_\\_init\\_\\_}
\textit{Konstruktor třídy Knight. Volá konstruktor třídy Piece

Args:
  color (Colors): barva figurky
  position ([int, int]): pozice figurky}
\item \textit{Metoda: copy}
\textit{Vytvoří kopii instance třídy Knight.

Returns:
  Knight: kopie instance třídy Knight}
\item \textit{Metoda: possibleMoves}
\textit{Vrátí seznam možných tahů pro jezdce.

Args:
  board (dict): šachovnice
  
Returns:
  List of [int,int]: seznam možných tahů  }
\end{itemize}
\subsection*{Soubor: Pawn.py}
\begin{itemize}
 \subsection*{Třída: Pawn}
\begin{itemize}
\item{Třída reprezentující pěšce v šachu. Dědí od třídy Piece.

Attributes:
  color (Colors): barva figurky
  position ([int,int]): pozice figurky
  symbol (str): symbol figurky
  value (int): hodnota figurky
  hasMoved (bool): True, pokud figurka už byla pohnuta, jinak False
  lastMoveWasDouble (bool): True, pokud poslední tah figurkou byl dvojitý, jinak False}
\item{Konstruktor třídy Pawn. Volá konstruktor třídy Piece.

Args:
  color (Colors): barva figurky
  position ([int, int]): pozice figurky}
\item{Vytvoří kopii instance třídy Pawn.

Returns:
  Pawn: kopie instance třídy Pawn}
\item{Zkontroluje, zda je možné provést tah pěšcem a provede ho.

Args:
  board (dict): šachovnice
  end ([int, int]): cílová pozice tahu}
\item{Vrátí seznam možných tahů pro pěšce.

Args:
  board (dict): šachovnice
  
Returns:
  List of [int,int]: seznam možných tahů}
\end{itemize}
\item \textit{Metoda: \\_\\_init\\_\\_}
\textit{Konstruktor třídy Pawn. Volá konstruktor třídy Piece.

Args:
  color (Colors): barva figurky
  position ([int, int]): pozice figurky}
\item \textit{Metoda: copy}
\textit{Vytvoří kopii instance třídy Pawn.

Returns:
  Pawn: kopie instance třídy Pawn}
\item \textit{Metoda: move}
\textit{Zkontroluje, zda je možné provést tah pěšcem a provede ho.

Args:
  board (dict): šachovnice
  end ([int, int]): cílová pozice tahu}
\item \textit{Metoda: possibleMoves}
\textit{Vrátí seznam možných tahů pro pěšce.

Args:
  board (dict): šachovnice
  
Returns:
  List of [int,int]: seznam možných tahů}
\end{itemize}
\subsection*{Soubor: Piece.py}
\begin{itemize}
 \subsection*{Třída: Piece}
\begin{itemize}
\item{Třída reprezentující jednu figurku na šachovnici.

Attributes:
  color (Colors): Barva figurky
  position ([int,int]): Pozice figurky
  hasMoved (bool): True, pokud figurka už byla pohnuta, jinak False
  lastMoveWasDouble (bool): True, pokud poslední tah figurkou byl dvojitý, jinak False
  value (int): Hodnota figurky}
\item{Konstruktor třídy Piece. Nastaví barvu a pozici figurky.

Args: 
    color (Colors): Barva figurky
    position ([int, int]): Pozice figurky}
\item{Metoda pro pohnutí figurky po šachovnici

Args:
    board (Board): Sachovnice, na kterou se tah províde
    end ([row, col]): Policko, kam se tah províde}
\item{Vrací řádek, na kterém se figurka nachází

Returns:
    int: Řádek figurky}
\item{Vrací sloupec, na kterém se figurka nachází

Returns:
    int: Sloupec figurky}
\item{Funkce pro vypsání všech možných tahů danou figurkou

Args:
    board (Board): Sachovnice, na kterou se tah províde

Returns:
    (list of [row, col]): Seznam všech možných tahů ve formátu}
\item{Vraci kopii objektu. Pouziva se, kdybychom chteli mit kopii objektu, bez toho, aby se menil puvodni objekt.

Returns:
    (Piece): Kopie objektu}
\end{itemize}
\item \textit{Metoda: \\_\\_init\\_\\_}
\textit{Konstruktor třídy Piece. Nastaví barvu a pozici figurky.

Args: 
    color (Colors): Barva figurky
    position ([int, int]): Pozice figurky}
\item \textit{Metoda: move}
\textit{Metoda pro pohnutí figurky po šachovnici

Args:
    board (Board): Sachovnice, na kterou se tah províde
    end ([row, col]): Policko, kam se tah províde}
\item \textit{Metoda: row}
\textit{Vrací řádek, na kterém se figurka nachází

Returns:
    int: Řádek figurky}
\item \textit{Metoda: col}
\textit{Vrací sloupec, na kterém se figurka nachází

Returns:
    int: Sloupec figurky}
\item \textit{Metoda: possibleMoves}
\textit{Funkce pro vypsání všech možných tahů danou figurkou

Args:
    board (Board): Sachovnice, na kterou se tah províde

Returns:
    (list of [row, col]): Seznam všech možných tahů ve formátu}
\item \textit{Metoda: copy}
\textit{Vraci kopii objektu. Pouziva se, kdybychom chteli mit kopii objektu, bez toho, aby se menil puvodni objekt.

Returns:
    (Piece): Kopie objektu}
\end{itemize}
\subsection*{Soubor: Queen.py}
\begin{itemize}
 \subsection*{Třída: Queen}
\begin{itemize}
\item{Třída reprezentující figurku královny v šachu. Dědí od třídy Piece.

Attributes:
  color (Colors): Barva figurky
  position ([int,int]): Pozice figurky
  hasMoved (bool): True, pokud figurka už byla pohnuta, jinak False
  lastMoveWasDouble (bool): True, pokud poslední tah figurkou byl dvojitý, jinak False
  value (int): Hodnota figurky
  symbol (str): Symbol figurky}
\item{Konstruktor třídy Queen. Volá konstruktor třídy Piece.
    }
\item{Vytvoří kopii instance třídy Queen.

Returns:
  Queen: kopie instance}
\item{Vrátí seznam možných tahů pro královnu. 

Args:
  board (dict): šachovnice
  
Returns:
  List of [int, int]: seznam možných tahů figurky}
\end{itemize}
\item \textit{Metoda: \\_\\_init\\_\\_}
\textit{Konstruktor třídy Queen. Volá konstruktor třídy Piece.
    }
\item \textit{Metoda: copy}
\textit{Vytvoří kopii instance třídy Queen.

Returns:
  Queen: kopie instance}
\item \textit{Metoda: possibleMoves}
\textit{Vrátí seznam možných tahů pro královnu. 

Args:
  board (dict): šachovnice
  
Returns:
  List of [int, int]: seznam možných tahů figurky}
\end{itemize}
\subsection*{Soubor: Rook.py}
\begin{itemize}
 \subsection*{Třída: Rook}
\begin{itemize}
\item{Třída reprezentující figurku věže v šachu. Dědí od třídy Piece.

Attributes:
  color (Colors): Barva figurky
  position ([int,int]): Pozice figurky
  hasMoved (bool): True, pokud figurka už byla pohnuta, jinak False
  lastMoveWasDouble (bool): True, pokud poslední tah figurkou byl dvojitý, jinak False
  value (int): Hodnota figurky
  symbol (str): Symbol figurky}
\item{Konstruktor třídy Rook. Volá konstruktor třídy Piece.

Args:
  color (Colors): Barva figurky
  position ([int,int]): Pozice figurky}
\item{Vytvoří kopii instance třídy Rook.

Returns:
  Rook: kopie instance}
\item{Vrátí seznam možných tahů pro věž.

Args:
  board (dict): šachovnice
  
Returns:
  List of [int, int]: seznam možných tahů figurky}
\end{itemize}
\item \textit{Metoda: \\_\\_init\\_\\_}
\textit{Konstruktor třídy Rook. Volá konstruktor třídy Piece.

Args:
  color (Colors): Barva figurky
  position ([int,int]): Pozice figurky}
\item \textit{Metoda: copy}
\textit{Vytvoří kopii instance třídy Rook.

Returns:
  Rook: kopie instance}
\item \textit{Metoda: possibleMoves}
\textit{Vrátí seznam možných tahů pro věž.

Args:
  board (dict): šachovnice
  
Returns:
  List of [int, int]: seznam možných tahů figurky}
\end{itemize}
\subsection*{Soubor: __init__.py}
\begin{itemize}
\end{itemize}
\section*{Složka: \_\_pycache\_\_}
\subsection*{Soubor: __init__.py}
\begin{itemize}
\end{itemize}
\section*{Složka: \_\_pycache\_\_}
\section*{Složka: chessTrackGame}
\subsection*{Soubor: ChessTrackGame.py}
\begin{itemize}
 \subsection*{Třída: ChessTrackGame}
\begin{itemize}
\item{Hra ChessTrackGame.

Attributes:
    board (ChessTrackGameBoard): Hrací deska
    currentPlayer (Enum Colors): Barva hráče, který je na tahu}
\item{Inicializace hry.}
\item{Vrácení hrací desky.

Args:
    color (Enum Colors): Barva hráče.

Returns:
    ChessTrackGameBoard: Hrací deska.}
\item{Provedení tahu.

Args:
    position ([int,int]]): Pozice, na kterou se má kámen umístit.
    color (Enum Colors): Barva kamene.
    rightClick (bool): True, pokud se jedná o pravé tlačítko myši.
    
Returns:
    bool: True, pokud je tah platný, jinak False.}
\item{Kontrola konce hry.

Returns:
    String: Který hráč vyhrál, pokud hra skončila, jinak None.}
\item{Vypsání hrací desky do terminálu.}
\end{itemize}
\item \textit{Metoda: \\_\\_init\\_\\_}
\textit{Inicializace hry.}
\item \textit{Metoda: getBoard}
\textit{Vrácení hrací desky.

Args:
    color (Enum Colors): Barva hráče.

Returns:
    ChessTrackGameBoard: Hrací deska.}
\item \textit{Metoda: makeMove}
\textit{Provedení tahu.

Args:
    position ([int,int]]): Pozice, na kterou se má kámen umístit.
    color (Enum Colors): Barva kamene.
    rightClick (bool): True, pokud se jedná o pravé tlačítko myši.
    
Returns:
    bool: True, pokud je tah platný, jinak False.}
\item \textit{Metoda: checkEnd}
\textit{Kontrola konce hry.

Returns:
    String: Který hráč vyhrál, pokud hra skončila, jinak None.}
\item \textit{Metoda: \\_\\_printToTerminal}
\textit{Vypsání hrací desky do terminálu.}
\end{itemize}
\subsection*{Soubor: ChessTrackGameBoard.py}
\begin{itemize}
 \subsection*{Třída: ChessTrackGameBoard}
\begin{itemize}
\item{Hrací deska hry ChessTrackGame.

Attributes:
    board (list): Hrací deska
    moves (int): Počet tahů}
\item{Inicializace hrací desky.}
\item{Inicializace hrací desky.}
\item{Otočení hrací desky.}
\item{Umístění kamene na desku.

Args:
    position ([int,int]]): Pozice, na kterou se má kámen umístit.
    color (Enum Colors): Barva kamene.}
\item{Provedení tahu.

Args:
    position ([int,int]]): Pozice, na kterou se má kámen umístit.
    color (Enum Colors): Barva kamene.}
\item{Vrátí seznam seznamů reprezentující hrací desku.

Returns:
    list: Seznam seznamů reprezentující hrací desku.}
\item{Kontrola konce hry.

Returns:
    str: Výsledek hry.}
\item{Kontrola výherce.

Returns:
    str: Výsledek hry.}
\end{itemize}
\item \textit{Metoda: \\_\\_init\\_\\_}
\textit{Inicializace hrací desky.}
\item \textit{Metoda: \\_\\_populateBoard}
\textit{Inicializace hrací desky.}
\item \textit{Metoda: \\_\\_spinBoard}
\textit{Otočení hrací desky.}
\item \textit{Metoda: \\_\\_placeStone}
\textit{Umístění kamene na desku.

Args:
    position ([int,int]]): Pozice, na kterou se má kámen umístit.
    color (Enum Colors): Barva kamene.}
\item \textit{Metoda: makeMove}
\textit{Provedení tahu.

Args:
    position ([int,int]]): Pozice, na kterou se má kámen umístit.
    color (Enum Colors): Barva kamene.}
\item \textit{Metoda: getListOfBoard}
\textit{Vrátí seznam seznamů reprezentující hrací desku.

Returns:
    list: Seznam seznamů reprezentující hrací desku.}
\item \textit{Metoda: checkEnd}
\textit{Kontrola konce hry.

Returns:
    str: Výsledek hry.}
\item \textit{Metoda: \\_\\_checkWinner}
\textit{Kontrola výherce.

Returns:
    str: Výsledek hry.}
\end{itemize}
\subsection*{Soubor: __init__.py}
\begin{itemize}
\end{itemize}
\section*{Složka: \_\_pycache\_\_}
\subsection*{Soubor: ChessWithFogOfWar.py}
\begin{itemize}
 \subsection*{Třída: ChessWithFogOfWar}
\begin{itemize}
\item{Třída ChessWithFogOfWar slouží k reprezentaci hry Šachy s mlhou války.
    }
\item{Konstruktor třídy ChessWithFogOfWar
        }
\item{Vrací zakrytou šachovnici

Args:   
    color (Enum Colors): Barva hráče na tahu

Returns:
    Array of Field: zakrytá šachovnice}
\end{itemize}
\item \textit{Metoda: \\_\\_init\\_\\_}
\textit{Konstruktor třídy ChessWithFogOfWar
        }
\item \textit{Metoda: getBoard}
\textit{Vrací zakrytou šachovnici

Args:   
    color (Enum Colors): Barva hráče na tahu

Returns:
    Array of Field: zakrytá šachovnice}
\end{itemize}
\section*{Složka: connectFour}
\subsection*{Soubor: ConnectFour.py}
\begin{itemize}
 \subsection*{Třída: ConnectFour}
\begin{itemize}
\item{Hrací deska pro hru ConnectFour 
    }
\item{Inicializace hry ConnectFour
        }
\item{Vrací hrací desku

Args:
    color (Enum, optional): Barva hráče. Defaults to None.

Returns:
    List of Struct: Hrací deska}
\item{Provedení tahu

Args:
    position ([int,int]): Pozice tahu
    color (Enum, optional): Barva hráče. Defaults to None.
    rightClick (bool, optional): True, pokud se jedná o pravé tlačítko myši. Defaults to False.

Returns:
    bool: True, pokud se tah podařil, jinak False}
\item{Zjištění konce hry

Returns:
    Enum: Barva vítěze}
\item{Výpis hrací desky do konzole
        }
\end{itemize}
\item \textit{Metoda: \\_\\_init\\_\\_}
\textit{Inicializace hry ConnectFour
        }
\item \textit{Metoda: getBoard}
\textit{Vrací hrací desku

Args:
    color (Enum, optional): Barva hráče. Defaults to None.

Returns:
    List of Struct: Hrací deska}
\item \textit{Metoda: makeMove}
\textit{Provedení tahu

Args:
    position ([int,int]): Pozice tahu
    color (Enum, optional): Barva hráče. Defaults to None.
    rightClick (bool, optional): True, pokud se jedná o pravé tlačítko myši. Defaults to False.

Returns:
    bool: True, pokud se tah podařil, jinak False}
\item \textit{Metoda: checkEnd}
\textit{Zjištění konce hry

Returns:
    Enum: Barva vítěze}
\item \textit{Metoda: \\_\\_printToTerminal}
\textit{Výpis hrací desky do konzole
        }
\end{itemize}
\subsection*{Soubor: ConnectFourBoard.py}
\begin{itemize}
 \subsection*{Třída: ConnectFourBoard}
\begin{itemize}
\item{Hrací deska pro hru ConnectFour 
    }
\item{Inicializace hry ConnectFour
        }
\item{Vytvoří string hrací desky na výpis do konzole
        }
\item{Inicializace herní desky
        }
\item{Provedení tahu

Args:
    position ([int,int]): Pozice tahu
    up (bool, optional): Směr zjišťování umístění. Defaults to False, tedy dolů.
    
Returns:
    [int,int]: Pozice, kam se má umístit kámen            }
\item{Provedení tahu

Args:
    position ([int,int]): Pozice tahu
    color (Enum): Barva hráče
    
Returns:
    bool: True, pokud se tah podařil, jinak False}
\item{Zjistí, zda je herní deska plná

Returns:
    bool: True, pokud je deska plná, jinak False}
\item{Zjistí, zda je konec hry

Args:
    position ([int,int]): Pozice tahu
    color (Enum): Barva hráče
    
Returns:
    bool: True, pokud je konec hry, jinak False}
\item{Zjistí, zda hráč vyhrál

Args:
    color (Enum): Barva hráče
    
Returns:
    bool: True, pokud hráč vyhrál, jinak False}
\item{Zjistí, zda hráč vyhrál horizontálně

Args:
    color (Enum): Barva hráče
    
Returns:
    bool: True, pokud hráč vyhrál horizontálně, jinak False}
\item{Zjistí, zda hráč vyhrál vertikálně

Args:
    color (Enum): Barva hráče
    
Returns:
    bool: True, pokud hráč vyhrál vertikálně, jinak False}
\item{Zjistí, zda hráč vyhrál diagonálně

Args:
    color (Enum): Barva hráče
    
Returns:
    bool: True, pokud hráč vyhrál diagonálně, jinak False}
\item{Vrací šachovnici jako list

Returns:
    List of Struct : List, kde každý řádek je list obsahující figury na daném řádku}
\end{itemize}
\item \textit{Metoda: \\_\\_init\\_\\_}
\textit{Inicializace hry ConnectFour
        }
\item \textit{Metoda: \\_\\_str\\_\\_}
\textit{Vytvoří string hrací desky na výpis do konzole
        }
\item \textit{Metoda: \\_\\_populateBoard}
\textit{Inicializace herní desky
        }
\item \textit{Metoda: findMove}
\textit{Provedení tahu

Args:
    position ([int,int]): Pozice tahu
    up (bool, optional): Směr zjišťování umístění. Defaults to False, tedy dolů.
    
Returns:
    [int,int]: Pozice, kam se má umístit kámen            }
\item \textit{Metoda: makeMove}
\textit{Provedení tahu

Args:
    position ([int,int]): Pozice tahu
    color (Enum): Barva hráče
    
Returns:
    bool: True, pokud se tah podařil, jinak False}
\item \textit{Metoda: isFull}
\textit{Zjistí, zda je herní deska plná

Returns:
    bool: True, pokud je deska plná, jinak False}
\item \textit{Metoda: checkEnd}
\textit{Zjistí, zda je konec hry

Args:
    position ([int,int]): Pozice tahu
    color (Enum): Barva hráče
    
Returns:
    bool: True, pokud je konec hry, jinak False}
\item \textit{Metoda: checkWin}
\textit{Zjistí, zda hráč vyhrál

Args:
    color (Enum): Barva hráče
    
Returns:
    bool: True, pokud hráč vyhrál, jinak False}
\item \textit{Metoda: checkHorizontal}
\textit{Zjistí, zda hráč vyhrál horizontálně

Args:
    color (Enum): Barva hráče
    
Returns:
    bool: True, pokud hráč vyhrál horizontálně, jinak False}
\item \textit{Metoda: checkVertical}
\textit{Zjistí, zda hráč vyhrál vertikálně

Args:
    color (Enum): Barva hráče
    
Returns:
    bool: True, pokud hráč vyhrál vertikálně, jinak False}
\item \textit{Metoda: checkDiagonal}
\textit{Zjistí, zda hráč vyhrál diagonálně

Args:
    color (Enum): Barva hráče
    
Returns:
    bool: True, pokud hráč vyhrál diagonálně, jinak False}
\item \textit{Metoda: getListOfBoard}
\textit{Vrací šachovnici jako list

Returns:
    List of Struct : List, kde každý řádek je list obsahující figury na daném řádku}
\end{itemize}
\subsection*{Soubor: __init__.py}
\begin{itemize}
\end{itemize}
\section*{Složka: \_\_pycache\_\_}
\subsection*{Soubor: Enums.py}
\begin{itemize}
 \subsection*{Třída: Field}
\begin{itemize}
\item{Třída reprezentující políčko na šachovnici
  }
\item{Enum pro typy figurek
  }
\item{Enum pro barvy
  }
\item{Vrací string reprezentaci barvy

Returns:
    String: reprezentovaná barva}
\item{Vrátí inverzní barvu

Returns:
    Enum Colors: inverzní barva}
\item{Vrátí další barvu

Returns:
    Enum Colors: další barva}
\end{itemize}
 \subsection*{Třída: Figures}
\begin{itemize}
\item{Třída reprezentující políčko na šachovnici
  }
\item{Enum pro typy figurek
  }
\item{Enum pro barvy
  }
\item{Vrací string reprezentaci barvy

Returns:
    String: reprezentovaná barva}
\item{Vrátí inverzní barvu

Returns:
    Enum Colors: inverzní barva}
\item{Vrátí další barvu

Returns:
    Enum Colors: další barva}
\end{itemize}
\item \textit{Atribut: PAWN}
\item \textit{Atribut: ROOK}
\item \textit{Atribut: BISHOP}
\item \textit{Atribut: KNIGHT}
\item \textit{Atribut: QUEEN}
\item \textit{Atribut: KING}
\item \textit{Atribut: X}
\item \textit{Atribut: O}
\item \textit{Atribut: FLAG}
\item \textit{Atribut: EXPLOSION}
\item \textit{Atribut: MINE}
\item \textit{Atribut: ONE}
\item \textit{Atribut: TWO}
\item \textit{Atribut: THREE}
\item \textit{Atribut: FOUR}
\item \textit{Atribut: FIVE}
\item \textit{Atribut: SIX}
\item \textit{Atribut: SEVEN}
\item \textit{Atribut: EIGHT}
\item \textit{Atribut: SHADOW}
\item \textit{Atribut: MOLE}
\item \textit{Atribut: LOGO}
 \subsection*{Třída: Colors}
\begin{itemize}
\item{Třída reprezentující políčko na šachovnici
  }
\item{Enum pro typy figurek
  }
\item{Enum pro barvy
  }
\item{Vrací string reprezentaci barvy

Returns:
    String: reprezentovaná barva}
\item{Vrátí inverzní barvu

Returns:
    Enum Colors: inverzní barva}
\item{Vrátí další barvu

Returns:
    Enum Colors: další barva}
\end{itemize}
\item \textit{Atribut: WHITE}
\item \textit{Atribut: BLACK}
\item \textit{Atribut: RED}
\item \textit{Atribut: GREEN}
\item \textit{Metoda: \\_\\_str\\_\\_}
\textit{Vrací string reprezentaci barvy

Returns:
    String: reprezentovaná barva}
\item \textit{Metoda: changeColor}
\textit{Vrátí inverzní barvu

Returns:
    Enum Colors: inverzní barva}
\item \textit{Metoda: changeColorFour}
\textit{Vrátí další barvu

Returns:
    Enum Colors: další barva}
\end{itemize}
\subsection*{Soubor: GameTemplate.py}
\begin{itemize}
 \subsection*{Třída: GameTemplate}
\begin{itemize}
\item{Třída GameTemplate slouží k reprezentaci šablony hry.
    }
\item{Konstruktor třídy GameTemplate
        }
\end{itemize}
\item \textit{Metoda: \\_\\_init\\_\\_}
\textit{Konstruktor třídy GameTemplate
        }
\end{itemize}
\subsection*{Soubor: GameTests.py}
\begin{itemize}
 \subsection*{Třída: GameTests}
\begin{itemize}
\item{Testy na hry
    }
\item{Testuje, zda se vytvoří hrací pole

Args:
    name (string): jméno hry
    game\\_class (game): třída hry}
\item{Testuje, zda se figurka dá vybrat

Args:
    name (string): jméno hry
    game\\_class (game): třída hry
    position ([int, int]): pozice figurky, se kterou se má pohnout
    color (Enum Colors): barva na tahu}
\item{Testuje, zda se figurka nedá vybrat

Args:
    name (string): jméno hry
    game\\_class (game): třída hry
    position ([int, int]): pozice figurky, se kterou se má pohnout
    color (Enum Colors): barva na tahu}
\item{Testuje, zda se figurka nedá vybrat

Args:
    name (string): jméno hry
    game\\_class (game): třída hry
    position ([int, int]): pozice figurky, se kterou se má pohnout
    color (Enum Colors): barva na tahu}
\item{Testuje, zda se pohyb provede

Args:
    name (string): jméno hry
    game\\_class (game): třída hry
    choose\\_position ([int, int]): pozice figurky, se kterou se má pohnout
    color (Enum Colors): barva na tahu
    move\\_position ([int, int]): pozice, kam se má figurka pohnout}
\item{Testuje, zda se pohyb nelze provést

Args:
    name (string): jméno hry
    game\\_class (game): třída hry
    choose\\_position ([int, int]): pozice figurky, se kterou se má pohnout
    color (Enum Colors): barva na tahu
    move\\_position ([int, int]): pozice, kam se má figurka pohnout}
\item{Testuje, zda se pohyb nelze provést

Args:
    name (string): jméno hry
    game\\_class (game): třída hry
    choose\\_position ([int, int]): pozice figurky, se kterou se má pohnout
    color (Enum Colors): barva na tahu
    move\\_position ([int, int]): pozice, kam se má figurka pohnout}
\item{Testuje, zda hra skončila

Args:
    name (string): jméno hry
    game\\_class (game): třída hry}
\item{Testuje, zda je definováno vybírání figurky

Args:
    name (string): jméno hry
    game\\_class (game): třída hry}
\item{Testuje, zda je definován mlhový efekt

Args:
    name (string): jméno hry
    game\\_class (game): třída hry}
\item{Testuje, zda je definován počet hráčů

Args:
    name (string): jméno hry
    game\\_class (game): třída hry}
\item{Testuje, zda se hra zahraje do konce

Args:
    name (string): jméno hry
    game\\_class (game): třída hry}
\end{itemize}
\item \textit{Metoda: testInitialBoard}
\textit{Testuje, zda se vytvoří hrací pole

Args:
    name (string): jméno hry
    game\\_class (game): třída hry}
\item \textit{Metoda: testChoosePiece}
\textit{Testuje, zda se figurka dá vybrat

Args:
    name (string): jméno hry
    game\\_class (game): třída hry
    position ([int, int]): pozice figurky, se kterou se má pohnout
    color (Enum Colors): barva na tahu}
\item \textit{Metoda: testChooseWrongPiece}
\textit{Testuje, zda se figurka nedá vybrat

Args:
    name (string): jméno hry
    game\\_class (game): třída hry
    position ([int, int]): pozice figurky, se kterou se má pohnout
    color (Enum Colors): barva na tahu}
\item \textit{Metoda: testChooseUnablePiece}
\textit{Testuje, zda se figurka nedá vybrat

Args:
    name (string): jméno hry
    game\\_class (game): třída hry
    position ([int, int]): pozice figurky, se kterou se má pohnout
    color (Enum Colors): barva na tahu}
\item \textit{Metoda: testMakeMove}
\textit{Testuje, zda se pohyb provede

Args:
    name (string): jméno hry
    game\\_class (game): třída hry
    choose\\_position ([int, int]): pozice figurky, se kterou se má pohnout
    color (Enum Colors): barva na tahu
    move\\_position ([int, int]): pozice, kam se má figurka pohnout}
\item \textit{Metoda: testMakeWrongMove}
\textit{Testuje, zda se pohyb nelze provést

Args:
    name (string): jméno hry
    game\\_class (game): třída hry
    choose\\_position ([int, int]): pozice figurky, se kterou se má pohnout
    color (Enum Colors): barva na tahu
    move\\_position ([int, int]): pozice, kam se má figurka pohnout}
\item \textit{Metoda: testMakeUnableMove}
\textit{Testuje, zda se pohyb nelze provést

Args:
    name (string): jméno hry
    game\\_class (game): třída hry
    choose\\_position ([int, int]): pozice figurky, se kterou se má pohnout
    color (Enum Colors): barva na tahu
    move\\_position ([int, int]): pozice, kam se má figurka pohnout}
\item \textit{Metoda: testCheckEnd}
\textit{Testuje, zda hra skončila

Args:
    name (string): jméno hry
    game\\_class (game): třída hry}
\item \textit{Metoda: testWithChoosingPiece}
\textit{Testuje, zda je definováno vybírání figurky

Args:
    name (string): jméno hry
    game\\_class (game): třída hry}
\item \textit{Metoda: testFog}
\textit{Testuje, zda je definován mlhový efekt

Args:
    name (string): jméno hry
    game\\_class (game): třída hry}
\item \textit{Metoda: testNumberOfPlayers}
\textit{Testuje, zda je definován počet hráčů

Args:
    name (string): jméno hry
    game\\_class (game): třída hry}
\item \textit{Metoda: testSimulateFullGame}
\textit{Testuje, zda se hra zahraje do konce

Args:
    name (string): jméno hry
    game\\_class (game): třída hry}
\end{itemize}
\section*{Složka: humanDoNotWorry}
\subsection*{Soubor: HumanDoNotWorry.py}
\begin{itemize}
 \subsection*{Třída: HumanDoNotWorry}
\begin{itemize}
\item{Třída pro hru člověče, nezlob se.
    }
\item{Konstruktor třídy hry člověče, nezlob se.
        }
\item{Metoda vrátí hrací desku.

Args:
    color (Enum Colors, optional): Barva hráče na tahu. Defaults to None.

Returns:
    List of list of field: Hrací deska}
\item{Metoda zvolí figurku, kterou se bude hrát.

Args:
    position (int): Pozice figurky
    color (Enum Colors, optional): Barva hráče. Defaults to None.

Returns:
    bool: True, pokud se podařilo zvolit figurku, jinak False}
\item{Metoda provede tah figurkou.

Args:
    position ([int,int]): Pozice figurky
    color (Enum Colors, optional): Barva hráče. Defaults to None.
    rightClick (bool, optional): True, pokud se jedná o pravé tlačítko myši. Defaults to False.

Returns:
    bool: True, pokud se podařilo provést tah, jinak False}
\item{Metoda provede standardní tah figurkou.

Args:
    row (int): řádek
    col (int): sloupec}
\item{Metoda umístí figurku do cíle.

Args:
    board (Board): Hrací deska
    piece (Piece): Figurka}
\item{Metoda zkontroluje, zda hra skončila.

Returns:
    string: Vrátí barvu hráče, který vyhrál, jinak None}
\item{Metoda hodí kostkou.

Returns:
    int: Hodnota kostky}
\item{Metoda vytiskne hrací desku na obrazovku.
        }
\end{itemize}
\item \textit{Metoda: \\_\\_init\\_\\_}
\textit{Konstruktor třídy hry člověče, nezlob se.
        }
\item \textit{Metoda: getBoard}
\textit{Metoda vrátí hrací desku.

Args:
    color (Enum Colors, optional): Barva hráče na tahu. Defaults to None.

Returns:
    List of list of field: Hrací deska}
\item \textit{Metoda: choosePiece}
\textit{Metoda zvolí figurku, kterou se bude hrát.

Args:
    position (int): Pozice figurky
    color (Enum Colors, optional): Barva hráče. Defaults to None.

Returns:
    bool: True, pokud se podařilo zvolit figurku, jinak False}
\item \textit{Metoda: makeMove}
\textit{Metoda provede tah figurkou.

Args:
    position ([int,int]): Pozice figurky
    color (Enum Colors, optional): Barva hráče. Defaults to None.
    rightClick (bool, optional): True, pokud se jedná o pravé tlačítko myši. Defaults to False.

Returns:
    bool: True, pokud se podařilo provést tah, jinak False}
\item \textit{Metoda: \\_\\_makeStandartMove}
\textit{Metoda provede standardní tah figurkou.

Args:
    row (int): řádek
    col (int): sloupec}
\item \textit{Metoda: \\_\\_placeToFinal}
\textit{Metoda umístí figurku do cíle.

Args:
    board (Board): Hrací deska
    piece (Piece): Figurka}
\item \textit{Metoda: checkEnd}
\textit{Metoda zkontroluje, zda hra skončila.

Returns:
    string: Vrátí barvu hráče, který vyhrál, jinak None}
\item \textit{Metoda: rollDice}
\textit{Metoda hodí kostkou.

Returns:
    int: Hodnota kostky}
\item \textit{Metoda: \\_\\_printToTerminal}
\textit{Metoda vytiskne hrací desku na obrazovku.
        }
\end{itemize}
\subsection*{Soubor: HumanDoNotWorryBoard.py}
\begin{itemize}
 \subsection*{Třída: HumanDoNotWorryBoard}
\begin{itemize}
\item{Třída HumanDoNotWorryBoard slouží k reprezentaci hrací desky hry člověče, nezlob se.
    }
\item{Konstruktor třídy HumanDoNotWorryBoard.
        }
\item{Vrátí textovou reprezentaci instance třídy HumanDoNotWorryBoard.
        }
\item{Metoda, která zjistí, zda je alespoň jedna figurka zadané barvy nasazena.

Args:
    color (Colors): Barva figurky}
\item{Metoda, která naplní hrací desku figurkami.
        }
\item{Metoda, která vrátí seznam seznamů reprezentující hrací desku.

Returns:
    List of List of Field: Seznam seznamů reprezentující hrací desku}
\end{itemize}
\item \textit{Metoda: \\_\\_init\\_\\_}
\textit{Konstruktor třídy HumanDoNotWorryBoard.
        }
\item \textit{Metoda: \\_\\_str\\_\\_}
\textit{Vrátí textovou reprezentaci instance třídy HumanDoNotWorryBoard.
        }
\item \textit{Metoda: isDeployed}
\textit{Metoda, která zjistí, zda je alespoň jedna figurka zadané barvy nasazena.

Args:
    color (Colors): Barva figurky}
\item \textit{Metoda: \\_\\_populateBoard}
\textit{Metoda, která naplní hrací desku figurkami.
        }
\item \textit{Metoda: getListOfBoard}
\textit{Metoda, která vrátí seznam seznamů reprezentující hrací desku.

Returns:
    List of List of Field: Seznam seznamů reprezentující hrací desku}
\end{itemize}
\section*{Složka: pieces}
\subsection*{Soubor: BlackPiece.py}
\begin{itemize}
 \subsection*{Třída: BlackPiece}
\begin{itemize}
\item{Třída BlackPiece slouží k reprezentaci jedné černé figurky.
    }
\item{Konstruktor třídy BlackPiece.

Args:
    position ([int,int]): Pozice figurky}
\item{Vrací textovou reprezentaci instance třídy BlackPiece.

Returns:
    str: Textová reprezentace instance třídy BlackPiece}
\end{itemize}
\item \textit{Metoda: \\_\\_init\\_\\_}
\textit{Konstruktor třídy BlackPiece.

Args:
    position ([int,int]): Pozice figurky}
\item \textit{Metoda: \\_\\_str\\_\\_}
\textit{Vrací textovou reprezentaci instance třídy BlackPiece.

Returns:
    str: Textová reprezentace instance třídy BlackPiece}
\end{itemize}
\subsection*{Soubor: GreenPiece.py}
\begin{itemize}
 \subsection*{Třída: GreenPiece}
\begin{itemize}
\item{Třída GreenPiece slouží k reprezentaci jedné zelené figurky.
    }
\item{Konstruktor třídy GreenPiece.

Args:
    position ([int,int]): Pozice figurky}
\item{Vrací textovou reprezentaci instance třídy GreenPiece.

Returns:
    str: Textová reprezentace instance třídy GreenPiece}
\end{itemize}
\item \textit{Metoda: \\_\\_init\\_\\_}
\textit{Konstruktor třídy GreenPiece.

Args:
    position ([int,int]): Pozice figurky}
\item \textit{Metoda: \\_\\_str\\_\\_}
\textit{Vrací textovou reprezentaci instance třídy GreenPiece.

Returns:
    str: Textová reprezentace instance třídy GreenPiece}
\end{itemize}
\subsection*{Soubor: Piece.py}
\begin{itemize}
 \subsection*{Třída: Piece}
\begin{itemize}
\item{Třída Piece slouží k reprezentaci jedné herní figurky.
    }
\item{Konstruktor třídy Piece.

Args:
    color (Colors): Barva figurky
    position ([int,int]): Pozice figurky}
\item{Metoda vrátí figurku domů.
        }
\item{Metoda vrátí seznam možných tahů figurky.

Args:
    number (int): Počet oček hoděných na kostce
    
Returns:
    List of [int,int]: Seznam možných tahů figurky}
\item{Metoda provede standardní tah figurkou.

Args:
    board (Board): Hrací deska
    position ([int,int]): Pozice figurky}
\end{itemize}
\item \textit{Metoda: \\_\\_init\\_\\_}
\textit{Konstruktor třídy Piece.

Args:
    color (Colors): Barva figurky
    position ([int,int]): Pozice figurky}
\item \textit{Metoda: returnHome}
\textit{Metoda vrátí figurku domů.
        }
\item \textit{Metoda: possibleMoves}
\textit{Metoda vrátí seznam možných tahů figurky.

Args:
    number (int): Počet oček hoděných na kostce
    
Returns:
    List of [int,int]: Seznam možných tahů figurky}
\item \textit{Metoda: makeStandartMove}
\textit{Metoda provede standardní tah figurkou.

Args:
    board (Board): Hrací deska
    position ([int,int]): Pozice figurky}
\end{itemize}
\subsection*{Soubor: RedPiece.py}
\begin{itemize}
 \subsection*{Třída: RedPiece}
\begin{itemize}
\item{Třída RedPiece slouží k reprezentaci jedné červené figurky.
    }
\item{Konstruktor třídy RedPiece.

Args:
    position ([int,int]): Pozice figurky}
\item{Vrací textovou reprezentaci instance třídy RedPiece.

Returns:
    str: Textová reprezentace instance třídy RedPiece}
\end{itemize}
\item \textit{Metoda: \\_\\_init\\_\\_}
\textit{Konstruktor třídy RedPiece.

Args:
    position ([int,int]): Pozice figurky}
\item \textit{Metoda: \\_\\_str\\_\\_}
\textit{Vrací textovou reprezentaci instance třídy RedPiece.

Returns:
    str: Textová reprezentace instance třídy RedPiece}
\end{itemize}
\subsection*{Soubor: WhitePiece.py}
\begin{itemize}
 \subsection*{Třída: WhitePiece}
\begin{itemize}
\item{Třída WhitePiece slouží k reprezentaci jedné bílé figurky.
    }
\item{Konstruktor třídy WhitePiece.

Args:
    position ([int,int]): Pozice figurky}
\item{Vrací textovou reprezentaci instance třídy WhitePiece.

Returns:
    str: Textová reprezentace instance třídy WhitePiece}
\end{itemize}
\item \textit{Metoda: \\_\\_init\\_\\_}
\textit{Konstruktor třídy WhitePiece.

Args:
    position ([int,int]): Pozice figurky}
\item \textit{Metoda: \\_\\_str\\_\\_}
\textit{Vrací textovou reprezentaci instance třídy WhitePiece.

Returns:
    str: Textová reprezentace instance třídy WhitePiece}
\end{itemize}
\subsection*{Soubor: __init__.py}
\begin{itemize}
\end{itemize}
\section*{Složka: \_\_pycache\_\_}
\subsection*{Soubor: __init__.py}
\begin{itemize}
\end{itemize}
\section*{Složka: \_\_pycache\_\_}
\subsection*{Soubor: ListOfGames.py}
\begin{itemize}
 \subsection*{Třída: Game}
\begin{itemize}
\item{Třída reprezentující hru v seznamu her pro frontend
    }
\item{Třída reprezentující seznam her pro frontend
    }
\item{Vrací seznam her pro frontend

Returns:
    List[Game]: seznam her}
\end{itemize}
 \subsection*{Třída: ListOfGames}
\begin{itemize}
\item{Třída reprezentující hru v seznamu her pro frontend
    }
\item{Třída reprezentující seznam her pro frontend
    }
\item{Vrací seznam her pro frontend

Returns:
    List[Game]: seznam her}
\end{itemize}
\item \textit{Metoda: getListOfGames}
\textit{Vrací seznam her pro frontend

Returns:
    List[Game]: seznam her}
\end{itemize}
\section*{Složka: mathGame}
\subsection*{Soubor: MathGame.py}
\begin{itemize}
 \subsection*{Třída: MathGame}
\begin{itemize}
\item{Třída reprezentující hru MathGame
    }
\item{Konstruktor třídy matematické hry. Vytvoří novou hru a nastaví počáteční hodnoty.
        }
\item{Vrátí šachovnici ve formě dvourozměrného pole objektů Field

Args:
    color (Enum Colors, optional): Barva hráče. Defaults to None.

Returns:
    list: dvourozměrné pole objektů Field}
\item{Vrátí možné tahy pro danou pozici

Args:
    position ([int,int]): pozice figury
    color (Enum Colors, optional): Barva figury. Defaults to None.

Returns:
    List of List of int: List možných tahů figury, prázdný list pokud tah není možný}
\item{Přesune figuru na jinou pozici

Args:
    move ([int,int]): nová pozice figury
    color (Enum Colors, optional): Barva figury. Defaults to None.
    rightClick (bool, optional): True, pokud hráč klikl pravým tlačítkem myši, jina False. Defaults to False.
    
Returns:
    bool: True, pokud se tah podařil, jinak False
    List of List of int: List možných tahů figury, pokud se dá pokračovat v pohybu}
\item{Zkontroluje, zda hra skončila

Returns:
    String: Vrací výsledek hry, pokud hra skončila, jinak None}
\item{Vytiskne aktuální stav hry na terminál
        }
\end{itemize}
\item \textit{Metoda: \\_\\_init\\_\\_}
\textit{Konstruktor třídy matematické hry. Vytvoří novou hru a nastaví počáteční hodnoty.
        }
\item \textit{Metoda: getBoard}
\textit{Vrátí šachovnici ve formě dvourozměrného pole objektů Field

Args:
    color (Enum Colors, optional): Barva hráče. Defaults to None.

Returns:
    list: dvourozměrné pole objektů Field}
\item \textit{Metoda: choosePiece}
\textit{Vrátí možné tahy pro danou pozici

Args:
    position ([int,int]): pozice figury
    color (Enum Colors, optional): Barva figury. Defaults to None.

Returns:
    List of List of int: List možných tahů figury, prázdný list pokud tah není možný}
\item \textit{Metoda: makeMove}
\textit{Přesune figuru na jinou pozici

Args:
    move ([int,int]): nová pozice figury
    color (Enum Colors, optional): Barva figury. Defaults to None.
    rightClick (bool, optional): True, pokud hráč klikl pravým tlačítkem myši, jina False. Defaults to False.
    
Returns:
    bool: True, pokud se tah podařil, jinak False
    List of List of int: List možných tahů figury, pokud se dá pokračovat v pohybu}
\item \textit{Metoda: checkEnd}
\textit{Zkontroluje, zda hra skončila

Returns:
    String: Vrací výsledek hry, pokud hra skončila, jinak None}
\item \textit{Metoda: \\_\\_printToTerminal}
\textit{Vytiskne aktuální stav hry na terminál
        }
\end{itemize}
\subsection*{Soubor: MathGameBoard.py}
\begin{itemize}
 \subsection*{Třída: MathGameBoard}
\begin{itemize}
\item{Třída reprezentující šachovnici pro matematickou hru. Šachovnice je 8x8 a obsahuje figury typu Colors a "TASK". Figury typu Colors jsou na šachovnici reprezentovány barvou, figury typu "TASK" jsou úkoly, které je potřeba splnit. 
    }
\item{Konstruktor třídy herní desky pro matematickou hru. Vytvoří novou šachovnici a nastaví počáteční hodnoty.
      }
\item{Vrací figuru na určeném místě na šachovnici, nebo None, pokud je políčko prázdné. 

Args:
    index [int, int]: Tuple dvou integerů, (row, col), oba 0-7}
\item{Nastaví políčko na šachovnici jako board[row,col] namísto board.board[row][col] 

Args:
    index ([int, int]): Tuple dvou integerů, (row, col), oba 0-7
    value (Piece): Instance třídy Piece, nebo None, pokud má být políčko prázdné

Returns:
    bool: True, pokud se podařilo nastavit políčko, jinak False}
\item{Vrací string reprezentaci šachovnice. Každé políčko je reprezentováno jako string, který je tvořen z informací o barvě a symbolu figury, nebo jako string "\\_\\_", pokud je políčko prázdné. Políčka jsou oddělena mezerou a jednotlivé řádky jsou odděleny znakem nového řádku (
).
      
      Returns:
          String: String reprezentace šachovnice
      }
\item{Nastaví šachovnici do počátečního stavu. Bílý je vpravo dole a černý vlevo nahoře.}
\item{Vrací počet zbývajících úkolů na šachovnici.

Returns:
    int: Počet zbývajících úkolů}
\item{Vrací pozici figury dané barvy.

Args:
    color (Enum Colors): Barva figurek, které chceme najít (Colors.WHITE nebo Colors.BLACK)

Returns: 
    [int, int]: Pozice figury na šachovnici}
\item{Vrací možné tahy figury na dané pozici.

Args:
    position ([int, int]): Pozice figury na šachovnici

Returns:
    List of [int, int]: List možných tahů figury}
\item{Přesune figuru na jiné místo na šachovnici.

Args:
    move ([int, int]): Nová pozice figury
    color (Enum Colors): Barva figury, kterou chceme přesunout}
\item{Vrací list všech figurek na šachovnici.

Returns: 
    List of Fields: List figurek na šachovnici}
\end{itemize}
\item \textit{Metoda: \\_\\_init\\_\\_}
\textit{Konstruktor třídy herní desky pro matematickou hru. Vytvoří novou šachovnici a nastaví počáteční hodnoty.
      }
\item \textit{Metoda: \\_\\_getitem\\_\\_}
\textit{Vrací figuru na určeném místě na šachovnici, nebo None, pokud je políčko prázdné. 

Args:
    index [int, int]: Tuple dvou integerů, (row, col), oba 0-7}
\item \textit{Metoda: \\_\\_setitem\\_\\_}
\textit{Nastaví políčko na šachovnici jako board[row,col] namísto board.board[row][col] 

Args:
    index ([int, int]): Tuple dvou integerů, (row, col), oba 0-7
    value (Piece): Instance třídy Piece, nebo None, pokud má být políčko prázdné

Returns:
    bool: True, pokud se podařilo nastavit políčko, jinak False}
\item \textit{Metoda: \\_\\_str\\_\\_}
\textit{Vrací string reprezentaci šachovnice. Každé políčko je reprezentováno jako string, který je tvořen z informací o barvě a symbolu figury, nebo jako string "\\_\\_", pokud je políčko prázdné. Políčka jsou oddělena mezerou a jednotlivé řádky jsou odděleny znakem nového řádku (
).
      
      Returns:
          String: String reprezentace šachovnice
      }
\item \textit{Metoda: \\_\\_populateBoard}
\textit{Nastaví šachovnici do počátečního stavu. Bílý je vpravo dole a černý vlevo nahoře.}
\item \textit{Metoda: TasksLeft}
\textit{Vrací počet zbývajících úkolů na šachovnici.

Returns:
    int: Počet zbývajících úkolů}
\item \textit{Metoda: getPosition}
\textit{Vrací pozici figury dané barvy.

Args:
    color (Enum Colors): Barva figurek, které chceme najít (Colors.WHITE nebo Colors.BLACK)

Returns: 
    [int, int]: Pozice figury na šachovnici}
\item \textit{Metoda: getPosibleMoves}
\textit{Vrací možné tahy figury na dané pozici.

Args:
    position ([int, int]): Pozice figury na šachovnici

Returns:
    List of [int, int]: List možných tahů figury}
\item \textit{Metoda: movePiece}
\textit{Přesune figuru na jiné místo na šachovnici.

Args:
    move ([int, int]): Nová pozice figury
    color (Enum Colors): Barva figury, kterou chceme přesunout}
\item \textit{Metoda: getListOfBoard}
\textit{Vrací list všech figurek na šachovnici.

Returns: 
    List of Fields: List figurek na šachovnici}
\end{itemize}
\subsection*{Soubor: __init__.py}
\begin{itemize}
\end{itemize}
\section*{Složka: \_\_pycache\_\_}
\section*{Složka: mines}
\subsection*{Soubor: Mines.py}
\begin{itemize}
 \subsection*{Třída: Mines}
\begin{itemize}
\item{Třída reprezentující hru Miny
    }
\item{Inicializace hry Miny
        }
\item{Provede tah hráče

Args:
    position ([int, int]): pozice, kam se má hráč pohnout
    color (Enum Colors): barva na tahu
    rightClick (bool): True, pokud hráč klikl pravým tlačítkem myši, jinak False
    
Returns:
    bool: úspěšnost tahu}
\item{Provede tah 

Args:
    position ([int, int]): pozice, kam se má hráč pohnout
    color (Enum Colors): barva na tahu
    
Returns:
    bool: úspěšnost tahu}
\item{Zkontroluje, zda hra skončila

Returns:
    None: pokud hra neskončila
    string: výsledek hry}
\item{Umístí vlajku na danou pozici

Args:
    position ([int, int]): pozice, kam se má vlajka umístit
    color (Enum Colors): barva na tahu
    
Returns:
    bool: úspěšnost umístění vlajky}
\item{Vrátí herní desku

Args:
    color (Enum Colors, optional): Barva, která je na tahu. Výchozí je None.

Returns:
    list: herní deska}
\item{Vypíše herní desku do konzole
        }
\end{itemize}
\item \textit{Metoda: \\_\\_init\\_\\_}
\textit{Inicializace hry Miny
        }
\item \textit{Metoda: makeMove}
\textit{Provede tah hráče

Args:
    position ([int, int]): pozice, kam se má hráč pohnout
    color (Enum Colors): barva na tahu
    rightClick (bool): True, pokud hráč klikl pravým tlačítkem myši, jinak False
    
Returns:
    bool: úspěšnost tahu}
\item \textit{Metoda: makeUncoverMove}
\textit{Provede tah 

Args:
    position ([int, int]): pozice, kam se má hráč pohnout
    color (Enum Colors): barva na tahu
    
Returns:
    bool: úspěšnost tahu}
\item \textit{Metoda: checkEnd}
\textit{Zkontroluje, zda hra skončila

Returns:
    None: pokud hra neskončila
    string: výsledek hry}
\item \textit{Metoda: placeFlag}
\textit{Umístí vlajku na danou pozici

Args:
    position ([int, int]): pozice, kam se má vlajka umístit
    color (Enum Colors): barva na tahu
    
Returns:
    bool: úspěšnost umístění vlajky}
\item \textit{Metoda: getBoard}
\textit{Vrátí herní desku

Args:
    color (Enum Colors, optional): Barva, která je na tahu. Výchozí je None.

Returns:
    list: herní deska}
\item \textit{Metoda: \\_\\_printToTerminal}
\textit{Vypíše herní desku do konzole
        }
\end{itemize}
\subsection*{Soubor: MinesBoard.py}
\begin{itemize}
 \subsection*{Třída: MinesBoard}
\begin{itemize}
\item{Třída reprezentující hrací desku hry Miny
    }
\item{Inicializace hrací desky Miny

Args:
    numberOfMines (int): počet min na hrací desce}
\item{Naplní hrací desku minami
        }
\item{Umístí miny na hrací desku
        }
\item{Spočítá miny kolem symbolu

Args:
    row (int): řádek
    col (int): sloupec
    
Returns:
    int: počet min kolem symbolu}
\item{Spočítá zbývající neoznačené miny na hrací desce

Returns:
    int: počet zbývajících neoznačených min}
\item{Spočítá počet vlajek na hrací desce

Returns:
    int: počet vlajek}
\item{Provede tah na hrací desce

Args:
    row (int): řádek
    col (int): sloupec
    
Returns:
    bool: True, pokud tah byl úspěšný, jinak False}
\item{Odkryje dostupné symboly na hrací desce po odkrytí políčka

Args:
    row (int): řádek
    col (int): sloupec}
\item{Umístí vlajku na hrací desku

Args:
    row (int): řádek
    col (int): sloupec
    
Returns:
    bool: False, pokud se vlajka neumístí
    int: -1, pokud se správná vlajka odstraní, 1, pokud se vlajka umístí správně a 0, pokud se vlajka odstraní nebo přidá na špatné místo}
\item{Vygeneruje textovou reprezentaci hrací desky

Returns:
    string: textová reprezentace hrací desky}
\end{itemize}
\item \textit{Metoda: \\_\\_init\\_\\_}
\textit{Inicializace hrací desky Miny

Args:
    numberOfMines (int): počet min na hrací desce}
\item \textit{Metoda: \\_\\_populateBoard}
\textit{Naplní hrací desku minami
        }
\item \textit{Metoda: \\_\\_placeMines}
\textit{Umístí miny na hrací desku
        }
\item \textit{Metoda: countMinesAroundSymbol}
\textit{Spočítá miny kolem symbolu

Args:
    row (int): řádek
    col (int): sloupec
    
Returns:
    int: počet min kolem symbolu}
\item \textit{Metoda: minesRemaining}
\textit{Spočítá zbývající neoznačené miny na hrací desce

Returns:
    int: počet zbývajících neoznačených min}
\item \textit{Metoda: flagsPlanted}
\textit{Spočítá počet vlajek na hrací desce

Returns:
    int: počet vlajek}
\item \textit{Metoda: makeMove}
\textit{Provede tah na hrací desce

Args:
    row (int): řádek
    col (int): sloupec
    
Returns:
    bool: True, pokud tah byl úspěšný, jinak False}
\item \textit{Metoda: \\_\\_showBoard}
\textit{Odkryje dostupné symboly na hrací desce po odkrytí políčka

Args:
    row (int): řádek
    col (int): sloupec}
\item \textit{Metoda: placeFlag}
\textit{Umístí vlajku na hrací desku

Args:
    row (int): řádek
    col (int): sloupec
    
Returns:
    bool: False, pokud se vlajka neumístí
    int: -1, pokud se správná vlajka odstraní, 1, pokud se vlajka umístí správně a 0, pokud se vlajka odstraní nebo přidá na špatné místo}
\item \textit{Metoda: \\_\\_str\\_\\_}
\textit{Vygeneruje textovou reprezentaci hrací desky

Returns:
    string: textová reprezentace hrací desky}
\end{itemize}
\subsection*{Soubor: __init__.py}
\begin{itemize}
\end{itemize}
\section*{Složka: \_\_pycache\_\_}
\section*{Složka: pexeso}
\subsection*{Soubor: Pexeso.py}
\begin{itemize}
 \subsection*{Třída: Pexeso}
\begin{itemize}
\item{Třída reprezentující hru Pexeso
    }
\item{Inicializace hry Pexeso
        }
\item{Vrátí hrací plochu

Args:
    color (Enum Colors): barva hráče

Returns:
    list: hrací plocha}
\item{Provede tah

Args:
    position ([int, int]): pozice, kterou chce hráč otočit
    color (Enum Colors): barva na tahu
    rightClick (bool): True, pokud hráč klikl pravým tlačítkem myši, jinak False
    
Returns:
    bool: úspěšnost tahu}
\item{Zkontroluje, zda hra skončila

Returns:
    None: pokud hra neskončila
    string: výsledek hry}
\item{Funkce pro vypsání hrací plochy do terminálu
        }
\end{itemize}
\item \textit{Metoda: \\_\\_init\\_\\_}
\textit{Inicializace hry Pexeso
        }
\item \textit{Metoda: getBoard}
\textit{Vrátí hrací plochu

Args:
    color (Enum Colors): barva hráče

Returns:
    list: hrací plocha}
\item \textit{Metoda: makeMove}
\textit{Provede tah

Args:
    position ([int, int]): pozice, kterou chce hráč otočit
    color (Enum Colors): barva na tahu
    rightClick (bool): True, pokud hráč klikl pravým tlačítkem myši, jinak False
    
Returns:
    bool: úspěšnost tahu}
\item \textit{Metoda: checkEnd}
\textit{Zkontroluje, zda hra skončila

Returns:
    None: pokud hra neskončila
    string: výsledek hry}
\item \textit{Metoda: \\_\\_printToTerminal}
\textit{Funkce pro vypsání hrací plochy do terminálu
        }
\end{itemize}
\subsection*{Soubor: PexesoBoard.py}
\begin{itemize}
 \subsection*{Třída: PexesoBoard}
\begin{itemize}
\item{Reprezentace hrací desky hry Pexeso
    }
\item{Inicializace hrací desky
        }
\item{Vygeneruje náhodné kartičky na hrací desku
        }
\item{Zamíchá kartičky na hrací desce
        }
\item{Schovej všechny kartičky
        }
\item{Vrátí textovou reprezentaci hrací desky
        }
\item{Provede tah
        }
\item{Vrátí, zda je kartička na dané pozici odhalena

Args:
    position ([int, int]): pozice kartičky
    
Returns:
    bool: zda je kartička odhalena}
\item{Vrátí, zda je kartička na dané pozici uhodnuta

Args:
    position ([int, int]): pozice kartičky
    
Returns:
    bool: zda je kartička uhodnuta}
\item{Vrátí hrací plochu
        }
\end{itemize}
\item \textit{Metoda: \\_\\_init\\_\\_}
\textit{Inicializace hrací desky
        }
\item \textit{Metoda: \\_\\_populateBoard}
\textit{Vygeneruje náhodné kartičky na hrací desku
        }
\item \textit{Metoda: shuffleBoard}
\textit{Zamíchá kartičky na hrací desce
        }
\item \textit{Metoda: hideCards}
\textit{Schovej všechny kartičky
        }
\item \textit{Metoda: \\_\\_str\\_\\_}
\textit{Vrátí textovou reprezentaci hrací desky
        }
\item \textit{Metoda: makeMove}
\textit{Provede tah
        }
\item \textit{Metoda: isRevealed}
\textit{Vrátí, zda je kartička na dané pozici odhalena

Args:
    position ([int, int]): pozice kartičky
    
Returns:
    bool: zda je kartička odhalena}
\item \textit{Metoda: isCompleted}
\textit{Vrátí, zda je kartička na dané pozici uhodnuta

Args:
    position ([int, int]): pozice kartičky
    
Returns:
    bool: zda je kartička uhodnuta}
\item \textit{Metoda: getListOfBoard}
\textit{Vrátí hrací plochu
        }
\end{itemize}
\subsection*{Soubor: PexesoCard.py}
\begin{itemize}
 \subsection*{Třída: PexesoCard}
\begin{itemize}
\item{Reprezentace jedné kartičky hry Pexeso
    }
\item{Inicializace kartičky

Args:
    symbol (Field): symbol kartičky
    odhalena (bool): zda je kartička odhalena}
\item{Vrátí symbol kartičky

Returns:
    Field: symbol kartičky}
\item{Porovná symboly dvou kartiček

Args:
    other (PexesoCard): druhá kartička

Returns:
    bool: zda jsou symboly stejné}
\item{Otočí kartičku
        }
\item{Schovej kartičku
        }
\item{Odhal kartičku
        }
\item{Označ kartičku jako vyřešenou
        }
\end{itemize}
\item \textit{Metoda: \\_\\_init\\_\\_}
\textit{Inicializace kartičky

Args:
    symbol (Field): symbol kartičky
    odhalena (bool): zda je kartička odhalena}
\item \textit{Metoda: getSymbol}
\textit{Vrátí symbol kartičky

Returns:
    Field: symbol kartičky}
\item \textit{Metoda: equals}
\textit{Porovná symboly dvou kartiček

Args:
    other (PexesoCard): druhá kartička

Returns:
    bool: zda jsou symboly stejné}
\item \textit{Metoda: turn}
\textit{Otočí kartičku
        }
\item \textit{Metoda: hide}
\textit{Schovej kartičku
        }
\item \textit{Metoda: reveal}
\textit{Odhal kartičku
        }
\item \textit{Metoda: match}
\textit{Označ kartičku jako vyřešenou
        }
\end{itemize}
\subsection*{Soubor: __init__.py}
\begin{itemize}
\end{itemize}
\section*{Složka: \_\_pycache\_\_}
\section*{Složka: ticTacToe}
\subsection*{Soubor: TicTacToe.py}
\begin{itemize}
 \subsection*{Třída: TicTacToe}
\begin{itemize}
\item{Třída reprezentující hru Piškvorky
    }
\item{Inicializace hry
        }
\item{Vrátí hrací desku

Args:
    color (Enum Colors, optional): barva hráče na tahu. Výchozí nastavení je na pravidelném střídání.

Returns:
    List: hrací deska}
\item{Metoda na zahrání tahu

Args:
    index ([int,int]): pozice kterou chce hráč obsadit
    player (Enum Colors, optional): Hráč co má být na tahu. Výchozí je nastaveno na střídání.
    rightClick (bool, optional): True, pokud se jedná o pravé tlačítko myši. Defaults to False.

Returns:
    Boolean: true pokud se tah podařil, jinak false}
\item{Kontroluje jestli hra skončila

Returns:
    String: "Draw" pokud je remíza
    String: "{barva} won" pokud někdo vyhrál
    None: pokud hra neskončila}
\item{Resetuje hru
        }
\item{Vytiskne hrací desku do konzole
        }
\item{Kontroluje jestli někdo vyhrál

Returns:
    Enum Colors: barva hráče, který vyhrál
    None: pokud nikdo nevyhrál}
\item{Kontroluje jestli je remíza

Returns:
    Boolean: true pokud je remíza, jinak false}
\end{itemize}
\item \textit{Metoda: \\_\\_init\\_\\_}
\textit{Inicializace hry
        }
\item \textit{Metoda: getBoard}
\textit{Vrátí hrací desku

Args:
    color (Enum Colors, optional): barva hráče na tahu. Výchozí nastavení je na pravidelném střídání.

Returns:
    List: hrací deska}
\item \textit{Metoda: makeMove}
\textit{Metoda na zahrání tahu

Args:
    index ([int,int]): pozice kterou chce hráč obsadit
    player (Enum Colors, optional): Hráč co má být na tahu. Výchozí je nastaveno na střídání.
    rightClick (bool, optional): True, pokud se jedná o pravé tlačítko myši. Defaults to False.

Returns:
    Boolean: true pokud se tah podařil, jinak false}
\item \textit{Metoda: checkEnd}
\textit{Kontroluje jestli hra skončila

Returns:
    String: "Draw" pokud je remíza
    String: "{barva} won" pokud někdo vyhrál
    None: pokud hra neskončila}
\item \textit{Metoda: reset}
\textit{Resetuje hru
        }
\item \textit{Metoda: \\_\\_printToTerminal}
\textit{Vytiskne hrací desku do konzole
        }
\item \textit{Metoda: \\_\\_checkWinner}
\textit{Kontroluje jestli někdo vyhrál

Returns:
    Enum Colors: barva hráče, který vyhrál
    None: pokud nikdo nevyhrál}
\item \textit{Metoda: \\_\\_checkDraw}
\textit{Kontroluje jestli je remíza

Returns:
    Boolean: true pokud je remíza, jinak false}
\end{itemize}
\subsection*{Soubor: TicTacToeBoard.py}
\begin{itemize}
 \subsection*{Třída: TicTacToeBoard}
\begin{itemize}
\item{Třída reprezentující hrací desku hry Piškvorky
    }
\item{Konstruktor třídy TicTacToeBoard
        }
\item{Inicializace herní desky
        }
\item{Vytiskne hrací desku do konzole
        }
\item{Vrací seznam políček na hrací desce

Returns:
    List of Struct Field: \\_description\\_}
\end{itemize}
\item \textit{Metoda: \\_\\_init\\_\\_}
\textit{Konstruktor třídy TicTacToeBoard
        }
\item \textit{Metoda: \\_\\_populateBoard}
\textit{Inicializace herní desky
        }
\item \textit{Metoda: \\_\\_str\\_\\_}
\textit{Vytiskne hrací desku do konzole
        }
\item \textit{Metoda: getListOfBoard}
\textit{Vrací seznam políček na hrací desce

Returns:
    List of Struct Field: \\_description\\_}
\end{itemize}
\subsection*{Soubor: __init__.py}
\begin{itemize}
\end{itemize}
\section*{Složka: \_\_pycache\_\_}
\subsection*{Soubor: __init__.py}
\begin{itemize}
\end{itemize}
\section*{Složka: \_\_pycache\_\_}
\subsection*{Soubor: GameView.py}
\begin{itemize}
 \subsection*{Třída: ClickableLabel}
\begin{itemize}
\item{Třída ClickableLabel slouží k vytvoření klikatelného labelu, který může vyslat signál o kliknutí na dané políčko
    }
\item{Konstruktor třídy ClickableLabel 
        }
\item{Metoda, která se zavolá při kliknutí na dané políčko
        }
\item{Třída GameView slouží k zobrazení hry na grafickém rozhraní pomocí PyQt5
    }
\item{Konstruktor

Args:
    game (game): Hra, kterou chceme zobrazit}
\item{Funkce pro nastavení obrázku figurky na dané pozici

Args:
    row (int): řádek
    col (int): sloupec
    filePath (string): cesta k obrázku}
\item{Funkce pro vytvoření herní desky
        }
\item{Funkce pro zobrazení otázky
        }
\item{Funkce pro zpracování odpovědi na otázku

Args:
    correct (bool): Byla odpověď správná?}
\item{Funkce pro obsluhu kliknutí na políčko

Args:
    row (int): řádek
    col (int): sloupec
    button (string): tlačítko, které bylo stisknuto}
\item{Funkce pro výběr figurky

Args:
    row (int): řádek
    col (int): sloupec}
\item{Funkce pro provedení tahu

Args:
    row (int): řádek
    col (int): sloupec}
\item{Funkce pro zvýraznění políčka

Args:
    row (int): řádek
    col (int): sloupec}
\item{Funkce pro aktualizaci herní desky

Args:
    isFirst (bool, optional): Je to první aktualizace?. Defaults to False.}
\item{Funkce pro odstranění herní desky
        }
\item{Funkce pro zobrazení dialogového okna s výsledkem hry

Args:
    message (string): Výsledek hry}
\item{Funkce pro výběr figurky, na kterou se má pesák změnit
        }
\end{itemize}
\item \textit{Atribut: clicked}
\item \textit{Metoda: \\_\\_init\\_\\_}
\textit{Konstruktor třídy ClickableLabel 
        }
\item \textit{Metoda: mousePressEvent}
\textit{Metoda, která se zavolá při kliknutí na dané políčko
        }
 \subsection*{Třída: GameView}
\begin{itemize}
\item{Třída ClickableLabel slouží k vytvoření klikatelného labelu, který může vyslat signál o kliknutí na dané políčko
    }
\item{Konstruktor třídy ClickableLabel 
        }
\item{Metoda, která se zavolá při kliknutí na dané políčko
        }
\item{Třída GameView slouží k zobrazení hry na grafickém rozhraní pomocí PyQt5
    }
\item{Konstruktor

Args:
    game (game): Hra, kterou chceme zobrazit}
\item{Funkce pro nastavení obrázku figurky na dané pozici

Args:
    row (int): řádek
    col (int): sloupec
    filePath (string): cesta k obrázku}
\item{Funkce pro vytvoření herní desky
        }
\item{Funkce pro zobrazení otázky
        }
\item{Funkce pro zpracování odpovědi na otázku

Args:
    correct (bool): Byla odpověď správná?}
\item{Funkce pro obsluhu kliknutí na políčko

Args:
    row (int): řádek
    col (int): sloupec
    button (string): tlačítko, které bylo stisknuto}
\item{Funkce pro výběr figurky

Args:
    row (int): řádek
    col (int): sloupec}
\item{Funkce pro provedení tahu

Args:
    row (int): řádek
    col (int): sloupec}
\item{Funkce pro zvýraznění políčka

Args:
    row (int): řádek
    col (int): sloupec}
\item{Funkce pro aktualizaci herní desky

Args:
    isFirst (bool, optional): Je to první aktualizace?. Defaults to False.}
\item{Funkce pro odstranění herní desky
        }
\item{Funkce pro zobrazení dialogového okna s výsledkem hry

Args:
    message (string): Výsledek hry}
\item{Funkce pro výběr figurky, na kterou se má pesák změnit
        }
\end{itemize}
\item \textit{Metoda: \\_\\_init\\_\\_}
\textit{Konstruktor

Args:
    game (game): Hra, kterou chceme zobrazit}
\item \textit{Metoda: set\\_piece\\_image}
\textit{Funkce pro nastavení obrázku figurky na dané pozici

Args:
    row (int): řádek
    col (int): sloupec
    filePath (string): cesta k obrázku}
\item \textit{Metoda: create\\_board}
\textit{Funkce pro vytvoření herní desky
        }
\item \textit{Metoda: show\\_question}
\textit{Funkce pro zobrazení otázky
        }
\item \textit{Metoda: handle\\_answer}
\textit{Funkce pro zpracování odpovědi na otázku

Args:
    correct (bool): Byla odpověď správná?}
\item \textit{Metoda: handle\\_square\\_click}
\textit{Funkce pro obsluhu kliknutí na políčko

Args:
    row (int): řádek
    col (int): sloupec
    button (string): tlačítko, které bylo stisknuto}
\item \textit{Metoda: choose\\_piece}
\textit{Funkce pro výběr figurky

Args:
    row (int): řádek
    col (int): sloupec}
\item \textit{Metoda: make\\_move}
\textit{Funkce pro provedení tahu

Args:
    row (int): řádek
    col (int): sloupec}
\item \textit{Metoda: highlight\\_square}
\textit{Funkce pro zvýraznění políčka

Args:
    row (int): řádek
    col (int): sloupec}
\item \textit{Metoda: update\\_board}
\textit{Funkce pro aktualizaci herní desky

Args:
    isFirst (bool, optional): Je to první aktualizace?. Defaults to False.}
\item \textit{Metoda: remove\\_board}
\textit{Funkce pro odstranění herní desky
        }
\item \textit{Metoda: game\\_ended}
\textit{Funkce pro zobrazení dialogového okna s výsledkem hry

Args:
    message (string): Výsledek hry}
\item \textit{Metoda: promote\\_pawn}
\textit{Funkce pro výběr figurky, na kterou se má pesák změnit
        }
\end{itemize}
\subsection*{Soubor: GetResource.py}
\begin{itemize}
 \subsection*{Třída: GetResource}
\begin{itemize}
\item{Třída GetResource slouží k získání cesty k obrázku, který reprezentuje daný zdroj.
    }
\item{Metoda na základě zadaného zdroje vrátí cestu k obrázku, který reprezentuje daný zdroj.

Args:
    resource (str): Zdroj, pro který chceme získat cestu k obrázku.

Returns:
    str: cesta k obrázku, který reprezentuje zadaný zdroj.}
\end{itemize}
\item \textit{Metoda: getResource}
\textit{Metoda na základě zadaného zdroje vrátí cestu k obrázku, který reprezentuje daný zdroj.

Args:
    resource (str): Zdroj, pro který chceme získat cestu k obrázku.

Returns:
    str: cesta k obrázku, který reprezentuje zadaný zdroj.}
\end{itemize}
\subsection*{Soubor: main.py}
\begin{itemize}
\end{itemize}
\subsection*{Soubor: MainView.py}
\begin{itemize}
 \subsection*{Třída: MainView}
\begin{itemize}
\item{Třída MainView slouží k zobrazení hlavního menu aplikace.
    }
\item{Konstruktor třídy
        }
\item{Spustí hru dle jména hry

Args:
    game (Game): objekt Game, který obsahuje název hry a objekt hry}
\end{itemize}
\item \textit{Metoda: \\_\\_init\\_\\_}
\textit{Konstruktor třídy
        }
\item \textit{Metoda: start\\_game}
\textit{Spustí hru dle jména hry

Args:
    game (Game): objekt Game, který obsahuje název hry a objekt hry}
\end{itemize}
\section*{Složka: questions}
\subsection*{Soubor: GenerateQuestion.py}
\begin{itemize}
 \subsection*{Třída: GenerateQuestion}
\begin{itemize}
\item{Třída na generování otázek
Pro generování otázek použij metodu generateQuestion
Pro kontrolu odpovědí použij metodu checkAnswer s parametrem answer
Pro získání správné odpovědi použij funkci doupovcuvOperator
Výsledky se zaokrouhlují na celá čísla!!!}
\item{Konstruktor třídy GenerateQuestion
        }
\item{Metoda na generování otázek

Args: 
    n (int): Číslo otázky, defaultně náhodné

Returns:
    string, string (questionText, questionLatex): otázka}
\end{itemize}
\item \textit{Metoda: \\_\\_init\\_\\_}
\textit{Konstruktor třídy GenerateQuestion
        }
\item \textit{Metoda: generateQuestion}
\textit{Metoda na generování otázek

Args: 
    n (int): Číslo otázky, defaultně náhodné

Returns:
    string, string (questionText, questionLatex): otázka}
\end{itemize}
\section*{Složka: generators}
\subsection*{Soubor: AnalyticGeometryQuestionGenerator.py}
\begin{itemize}
 \subsection*{Třída: AnalyticGeometryQuestionGenerator}
\begin{itemize}
\item{Generátor otázek na analytickou geometrii
    }
\item{Konstruktor třídy AnalyticGeometryQuestionGenerator
        }
\item{Generování náhodné otázky na analytickou geometrii

Args:
    n (int): Číslo otázky, defaultně náhodné

Returns:
    AnalyticGeometryQuestionGenerator: Vrací samo sebe s vygenerovanou otázkou a odpovědí}
\end{itemize}
\item \textit{Metoda: \\_\\_init\\_\\_}
\textit{Konstruktor třídy AnalyticGeometryQuestionGenerator
        }
\item \textit{Metoda: generateQuestion}
\textit{Generování náhodné otázky na analytickou geometrii

Args:
    n (int): Číslo otázky, defaultně náhodné

Returns:
    AnalyticGeometryQuestionGenerator: Vrací samo sebe s vygenerovanou otázkou a odpovědí}
\end{itemize}
\subsection*{Soubor: DerivativeQuestionGenerator.py}
\begin{itemize}
 \subsection*{Třída: DerivativeQuestionGenerator}
\begin{itemize}
\item{Generátor otázek na derivace
    }
\item{Konstruktor třídy DerivativeQuestionGenerator
        }
\item{Generování náhodného polynomu

Args:
    degree (int): stupeň polynomu
    
Returns:
    sympy symbol: polynom }
\item{Generování náhodné otázky na derivace

Args:
    n (int): Číslo otázky, defaultně náhodné
    
Returns:
    DerivativeQuestionGenerator: Vrací samo sebe s vygenerovanou otázkou a odpovědí}
\end{itemize}
\item \textit{Metoda: \\_\\_init\\_\\_}
\textit{Konstruktor třídy DerivativeQuestionGenerator
        }
\item \textit{Metoda: generatePolynomial}
\textit{Generování náhodného polynomu

Args:
    degree (int): stupeň polynomu
    
Returns:
    sympy symbol: polynom }
\item \textit{Metoda: generateQuestion}
\textit{Generování náhodné otázky na derivace

Args:
    n (int): Číslo otázky, defaultně náhodné
    
Returns:
    DerivativeQuestionGenerator: Vrací samo sebe s vygenerovanou otázkou a odpovědí}
\end{itemize}
\subsection*{Soubor: FractionQuestionGenerator.py}
\begin{itemize}
 \subsection*{Třída: FractionQuestionGenerator}
\begin{itemize}
\item{Generátor otázek na zlomky
    }
\item{Konstruktor třídy FractionQuestionGenerator
        }
\item{Vygenerování náhodné dvojice čísel

Returns:
    int, int: (numerator, denominator), kde numerator je čitatel v rozmezí 10-100 a denominator jmenovatel v rozmezí 1-10}
\item{Zjednodušení zlomku

Args:
    numerator (int): čitatel 
    denominator (int): jmenovatel
Returns:
    int, int: (numerator, denominator), kde numerator je čitatel a denominator jmenovatel zjednodušeného zlomku}
\item{Výpočet nejmenšího společného násobku

Args:
    a (int): první číslo
    b (int): druhé číslo

Returns:
    int: Nejmenší společný násobek}
\item{Výpočet největšího společného dělitele

Args:
    a (int): první číslo
    b (int): druhé číslo
Returns:
    int: Největší společný dělitel}
\item{Převedení zlomku na string

Args:
    numerator (int): čitatel 
    denominator (int): jmenovatel
    
Returns:
    String: reprezentace zlomku v latexu}
\item{Převedení zlomku na string

Args:
    numerator (int): čitatel 
    denominator (int): jmenovatel
    
Returns:
    String: reprezentace zlomku ve formátu "a/b"}
\item{Vygenerování náhodné otázky na zlomky

Args:
    n (int): Číslo otázky, defaultně náhodné

Returns:
    FractionQuestionGenerator: Samo sebe s vygenerovanou otázkou a odpovědí}
\end{itemize}
\item \textit{Metoda: \\_\\_init\\_\\_}
\textit{Konstruktor třídy FractionQuestionGenerator
        }
\item \textit{Metoda: generateFraction}
\textit{Vygenerování náhodné dvojice čísel

Returns:
    int, int: (numerator, denominator), kde numerator je čitatel v rozmezí 10-100 a denominator jmenovatel v rozmezí 1-10}
\item \textit{Metoda: simplifyFraction}
\textit{Zjednodušení zlomku

Args:
    numerator (int): čitatel 
    denominator (int): jmenovatel
Returns:
    int, int: (numerator, denominator), kde numerator je čitatel a denominator jmenovatel zjednodušeného zlomku}
\item \textit{Metoda: lowestCommonMultiple}
\textit{Výpočet nejmenšího společného násobku

Args:
    a (int): první číslo
    b (int): druhé číslo

Returns:
    int: Nejmenší společný násobek}
\item \textit{Metoda: greatestCommonDivisor}
\textit{Výpočet největšího společného dělitele

Args:
    a (int): první číslo
    b (int): druhé číslo
Returns:
    int: Největší společný dělitel}
\item \textit{Metoda: fractionToString}
\textit{Převedení zlomku na string

Args:
    numerator (int): čitatel 
    denominator (int): jmenovatel
    
Returns:
    String: reprezentace zlomku v latexu}
\item \textit{Metoda: fractionToAnswer}
\textit{Převedení zlomku na string

Args:
    numerator (int): čitatel 
    denominator (int): jmenovatel
    
Returns:
    String: reprezentace zlomku ve formátu "a/b"}
\item \textit{Metoda: generateQuestion}
\textit{Vygenerování náhodné otázky na zlomky

Args:
    n (int): Číslo otázky, defaultně náhodné

Returns:
    FractionQuestionGenerator: Samo sebe s vygenerovanou otázkou a odpovědí}
\end{itemize}
\subsection*{Soubor: InfinitiveSeriesQuestionGenerator.py}
\begin{itemize}
 \subsection*{Třída: InfinitiveSeriesQuestionGenerator}
\begin{itemize}
\item{Generátor otázek na konvergenci nekonečných řad
    }
\item{Konstruktor třídy nekonečných řad
        }
\item{Generování náhodné otázky na konvergenci nekonečných řad

Args:
    n (int): Číslo otázky, defaultně náhodné

Returns:
    InfinitiveSeriesQuestionGenerator: Vrací samo sebe s vygenerovanou otázkou a odpovědí}
\end{itemize}
\item \textit{Metoda: \\_\\_init\\_\\_}
\textit{Konstruktor třídy nekonečných řad
        }
\item \textit{Metoda: generateQuestion}
\textit{Generování náhodné otázky na konvergenci nekonečných řad

Args:
    n (int): Číslo otázky, defaultně náhodné

Returns:
    InfinitiveSeriesQuestionGenerator: Vrací samo sebe s vygenerovanou otázkou a odpovědí}
\end{itemize}
\subsection*{Soubor: IntegralQuestionGenerator.py}
\begin{itemize}
 \subsection*{Třída: IntegralQuestionGenerator}
\begin{itemize}
\item{Generátor otázek na určení hodnoty integrálu
    }
\item{Konstruktor třídy IntegralQuestionGenerator
        }
\item{Generování náhodné otázky na určení hodnoty integrálu

Args:
    n (int): Číslo otázky, defaultně náhodné

Returns:
    IntegralQuestionGenerator: Vrací samo sebe s vygenerovanou otázkou a odpovědí}
\end{itemize}
\item \textit{Metoda: \\_\\_init\\_\\_}
\textit{Konstruktor třídy IntegralQuestionGenerator
        }
\item \textit{Metoda: generateQuestion}
\textit{Generování náhodné otázky na určení hodnoty integrálu

Args:
    n (int): Číslo otázky, defaultně náhodné

Returns:
    IntegralQuestionGenerator: Vrací samo sebe s vygenerovanou otázkou a odpovědí}
\end{itemize}
\subsection*{Soubor: KardinalNumberQuestionGenerator.py}
\begin{itemize}
 \subsection*{Třída: KardinalNumberQuestionGenerator}
\begin{itemize}
\item{Generátor otázek na kardinální čísla
    }
\item{Konstruktor třídy otázek na kardinální čísla
        }
\item{Generování náhodné otázky na kardinální čísla

Args:
    n (int): Číslo otázky, defaultně náhodné
    
Returns:
    KardinalNumberQuestionGenerator: Vrací samo sebe s vygenerovanou otázkou a odpovědí}
\end{itemize}
\item \textit{Metoda: \\_\\_init\\_\\_}
\textit{Konstruktor třídy otázek na kardinální čísla
        }
\item \textit{Metoda: generateQuestion}
\textit{Generování náhodné otázky na kardinální čísla

Args:
    n (int): Číslo otázky, defaultně náhodné
    
Returns:
    KardinalNumberQuestionGenerator: Vrací samo sebe s vygenerovanou otázkou a odpovědí}
\end{itemize}
\subsection*{Soubor: LinearEquationSystemQuestionGenerator.py}
\begin{itemize}
 \subsection*{Třída: LinearEquationSystemQuestionGenerator}
\begin{itemize}
\item{Generátor otázek na lineární soustavy rovnic
    }
\item{Konstruktor třídy soustav lineárních rovnic
        }
\item{Vygenerování lineární soustavy rovnic

Args:
    num\\_equations (int): počet chtěných rovnic
    num\\_variables (int): počet chtěných proměnných

Returns:
    sympy equations, sympy symbols: Vygenerované rovnice a jejich proměnné}
\item{Metoda na konverzi rovnic do latexu

Args:
    equations (sympy equations): vstupní rovnice

Returns:
    string: latexový zápis rovnic ve formátu string}
\item{Metoda na generování otázky

Args:
    n (int): Číslo otázky, defaultně náhodné

Returns:
    LinearEquationSystemQuestionGenerator: Funkce vrací sebe sama s vygenerovanou otázkou}
\end{itemize}
\item \textit{Metoda: \\_\\_init\\_\\_}
\textit{Konstruktor třídy soustav lineárních rovnic
        }
\item \textit{Metoda: generateLinearEquationSystem}
\textit{Vygenerování lineární soustavy rovnic

Args:
    num\\_equations (int): počet chtěných rovnic
    num\\_variables (int): počet chtěných proměnných

Returns:
    sympy equations, sympy symbols: Vygenerované rovnice a jejich proměnné}
\item \textit{Metoda: convert\\_to\\_latex}
\textit{Metoda na konverzi rovnic do latexu

Args:
    equations (sympy equations): vstupní rovnice

Returns:
    string: latexový zápis rovnic ve formátu string}
\item \textit{Metoda: generateQuestion}
\textit{Metoda na generování otázky

Args:
    n (int): Číslo otázky, defaultně náhodné

Returns:
    LinearEquationSystemQuestionGenerator: Funkce vrací sebe sama s vygenerovanou otázkou}
\end{itemize}
\subsection*{Soubor: MatrixQuestionGenerator.py}
\begin{itemize}
 \subsection*{Třída: MatrixQuestionGenerator}
\begin{itemize}
\item{Generátor otázek na matice
    }
\item{Konstruktor třídy matice
        }
\item{Generování regulární matice

Args:
    n (int): Velikost matice, defaultně náhodně mezi 2 a 4

Returns:
    numpy array int: Náhodná regulární matice}
\item{Výpočet determinantu matice

Args:
    matrix (numpy array int):  matice, ve tvaru numpy array intů
    
Returns:
    float: Hodnota determinantu matice}
\item{Výpočet inverzní matice

Args:
    matrix (numpy array int): matice

Returns:
    numpy array int: inverzní matice}
\item{Výpočet řádu matice

Args:
    matrix (numpy array int): matice

Returns:
    int: řád matice}
\item{Výpočet vlastních čísel matice

Args:
    matrix (numpy array int): matice

Returns:
    float: součet vlastní čísla matice}
\item{Metoda pro vygenerování matice ve formátu LaTeX

Args:
    matrix (numpy.ndarray): matice

Returns:
    string: matice ve formátu LaTeX}
\item{Metoda na generování otázek na matice

Returns:
    MatrixQuestionGenerator: Vrací sebe sama s vygenerovanými otázkami}
\end{itemize}
\item \textit{Metoda: \\_\\_init\\_\\_}
\textit{Konstruktor třídy matice
        }
\item \textit{Metoda: generateRegularMatrix}
\textit{Generování regulární matice

Args:
    n (int): Velikost matice, defaultně náhodně mezi 2 a 4

Returns:
    numpy array int: Náhodná regulární matice}
\item \textit{Metoda: calculateDeterminant}
\textit{Výpočet determinantu matice

Args:
    matrix (numpy array int):  matice, ve tvaru numpy array intů
    
Returns:
    float: Hodnota determinantu matice}
\item \textit{Metoda: calculateInverseMatrix}
\textit{Výpočet inverzní matice

Args:
    matrix (numpy array int): matice

Returns:
    numpy array int: inverzní matice}
\item \textit{Metoda: calculateRank}
\textit{Výpočet řádu matice

Args:
    matrix (numpy array int): matice

Returns:
    int: řád matice}
\item \textit{Metoda: calculateEigenvalues}
\textit{Výpočet vlastních čísel matice

Args:
    matrix (numpy array int): matice

Returns:
    float: součet vlastní čísla matice}
\item \textit{Metoda: getLatexMatrix}
\textit{Metoda pro vygenerování matice ve formátu LaTeX

Args:
    matrix (numpy.ndarray): matice

Returns:
    string: matice ve formátu LaTeX}
\item \textit{Metoda: generateQuestion}
\textit{Metoda na generování otázek na matice

Returns:
    MatrixQuestionGenerator: Vrací sebe sama s vygenerovanými otázkami}
\end{itemize}
\subsection*{Soubor: OrdinalNumberQuestionGenerator.py}
\begin{itemize}
 \subsection*{Třída: OrdinalNumberQuestionGenerator}
\begin{itemize}
\item{Generátor otázek na ordinální čísla
    }
\item{Konstruktor třídy otázek na ordinální čísla
        }
\item{Generování náhodné otázky na uspořádaná čísla

Args:
    n (int): Číslo otázky, defaultně náhodné

Returns:
    OrdinalNumberQuestionGenerator: Vrací samo sebe s vygenerovanou otázkou a odpovědí}
\end{itemize}
\item \textit{Metoda: \\_\\_init\\_\\_}
\textit{Konstruktor třídy otázek na ordinální čísla
        }
\item \textit{Metoda: generateQuestion}
\textit{Generování náhodné otázky na uspořádaná čísla

Args:
    n (int): Číslo otázky, defaultně náhodné

Returns:
    OrdinalNumberQuestionGenerator: Vrací samo sebe s vygenerovanou otázkou a odpovědí}
\end{itemize}
\subsection*{Soubor: RegularLanguageQuestionGenerator.py}
\begin{itemize}
 \subsection*{Třída: RegularLanguageQuestionGenerator}
\begin{itemize}
\item{Generátor otázek na regulární jazyky
    }
\item{Konstruktor třídy regulárních jazyků
        }
\item{Generování náhodné otázky na regulární jazyky

Args:
    n (int): Číslo otázky, defaultně náhodné

Returns:
    RegularLanguageQuestionGenerator: Vrací samo sebe s vygenerovanou otázkou a odpovědí}
\end{itemize}
\item \textit{Metoda: \\_\\_init\\_\\_}
\textit{Konstruktor třídy regulárních jazyků
        }
\item \textit{Metoda: generateQuestion}
\textit{Generování náhodné otázky na regulární jazyky

Args:
    n (int): Číslo otázky, defaultně náhodné

Returns:
    RegularLanguageQuestionGenerator: Vrací samo sebe s vygenerovanou otázkou a odpovědí}
\end{itemize}
\subsection*{Soubor: SetQuestionGenerator.py}
\begin{itemize}
 \subsection*{Třída: SetQuestionGenerator}
\begin{itemize}
\item{Generátor otázek na množiny
    }
\item{Konstruktor třídy otázek na množiny
        }
\item{Generování náhodné otázky na množiny

Args:
    n (int): Číslo otázky, defaultně náhodné

Returns:
    SetQuestionGenerator: Vrací samo sebe s vygenerovanou otázkou a odpovědí}
\end{itemize}
\item \textit{Metoda: \\_\\_init\\_\\_}
\textit{Konstruktor třídy otázek na množiny
        }
\item \textit{Metoda: generateQuestion}
\textit{Generování náhodné otázky na množiny

Args:
    n (int): Číslo otázky, defaultně náhodné

Returns:
    SetQuestionGenerator: Vrací samo sebe s vygenerovanou otázkou a odpovědí}
\end{itemize}
\subsection*{Soubor: __init__.py}
\begin{itemize}
\end{itemize}
\section*{Složka: \_\_pycache\_\_}
\subsection*{Soubor: Question.py}
\begin{itemize}
 \subsection*{Třída: Question}
\begin{itemize}
\item{Třída předpisu otázky
    }
\item{Konstruktor třídy Question
        }
\item{Metoda na výpis otázky s odpovědí

Returns:
    string: Vrátí otázku a odpověď }
\item{Metoda na kontrolu odpovědi

Args:
    string: answer - zadaná odpověď
    
Returns:
    bool: true pokud je odpověď správná}
\item{Metoda na získání odpovědi

Returns:
    string: odpověď}
\item{Funkce na generování otázky

Returns:
    string, string: Vrací otázku}
\end{itemize}
\item \textit{Metoda: \\_\\_init\\_\\_}
\textit{Konstruktor třídy Question
        }
\item \textit{Metoda: \\_\\_str\\_\\_}
\textit{Metoda na výpis otázky s odpovědí

Returns:
    string: Vrátí otázku a odpověď }
\item \textit{Metoda: checkAnswer}
\textit{Metoda na kontrolu odpovědi

Args:
    string: answer - zadaná odpověď
    
Returns:
    bool: true pokud je odpověď správná}
\item \textit{Metoda: doupovcuvOperator}
\textit{Metoda na získání odpovědi

Returns:
    string: odpověď}
\item \textit{Metoda: generateQuestion}
\textit{Funkce na generování otázky

Returns:
    string, string: Vrací otázku}
\end{itemize}
\subsection*{Soubor: QuestionTests.py}
\begin{itemize}
 \subsection*{Třída: QuestionTests}
\begin{itemize}
\item{Testy na generátory otázek
    }
\item{Testuje, zda se generuje nenulový počet otázek

Args:
    name (string): jméno generátoru
    generator\\_class (Question): třída generátoru}
\item{Testuje, zda se vygeneruje otázka

Args:
    name (string): jméno generátoru
    generator\\_class (Question): třída generátoru}
\item{Testuje, zda se vygeneruje odpověď

Args:
    name (string): jméno generátoru
    generator\\_class (Question): třída generátoru}
\item{Testuje, zda je odpověď správná

Args:
    name (string): jméno generátoru
    generator\\_class (Question): třída generátoru}
\item{Testuje, zda je odpověď špatná

Args:
    name (string): jméno generátoru
    generator\\_class (Question): třída generátoru}
\end{itemize}
\item \textit{Atribut: allClasses}
\item \textit{Metoda: testNumberOfQuestions}
\textit{Testuje, zda se generuje nenulový počet otázek

Args:
    name (string): jméno generátoru
    generator\\_class (Question): třída generátoru}
\item \textit{Metoda: testGenerateQuestion}
\textit{Testuje, zda se vygeneruje otázka

Args:
    name (string): jméno generátoru
    generator\\_class (Question): třída generátoru}
\item \textit{Metoda: testDoupovcuvOperator}
\textit{Testuje, zda se vygeneruje odpověď

Args:
    name (string): jméno generátoru
    generator\\_class (Question): třída generátoru}
\item \textit{Metoda: testCheckAnswer}
\textit{Testuje, zda je odpověď správná

Args:
    name (string): jméno generátoru
    generator\\_class (Question): třída generátoru}
\item \textit{Metoda: testWrongAnswer}
\textit{Testuje, zda je odpověď špatná

Args:
    name (string): jméno generátoru
    generator\\_class (Question): třída generátoru}
\item \textit{Metoda: testTimeSet}
\item \textit{Metoda: testHoderova}
\end{itemize}
\subsection*{Soubor: __init__.py}
\begin{itemize}
\end{itemize}
\section*{Složka: \_\_pycache\_\_}
\subsection*{Soubor: QuestionView.py}
\begin{itemize}
 \subsection*{Třída: MathQuestion}
\begin{itemize}
\item{Třída MathQuestion slouží k zobrazení matematické otázky.
    }
\item{Konstruktor třídy

Args:
    question (Question): Otázka, která se má zobrazit
    color (Colors): Barva hráče, který má odpovídat
    fullscreen (bool): Zda se má zobrazit na celou obrazovku
    callback (function): Funkce, která se má zavolat po zodpovězení otázky}
\item{Metoda na aktualizaci časovače
        }
\item{Metoda na oznámení, že čas vypršel
        }
\item{Metoda na kontrolu odpovědi
        }
\item{Metoda pro ukončení okna galantní cestou
        }
\item{Metoda na vytvoření rovnic ve formátu LaTeX

Args:
    latexEq (string): rovnice, které se mají vykreslit v LaTeXu}
\end{itemize}
\item \textit{Metoda: \\_\\_init\\_\\_}
\textit{Konstruktor třídy

Args:
    question (Question): Otázka, která se má zobrazit
    color (Colors): Barva hráče, který má odpovídat
    fullscreen (bool): Zda se má zobrazit na celou obrazovku
    callback (function): Funkce, která se má zavolat po zodpovězení otázky}
\item \textit{Metoda: update\\_timer}
\textit{Metoda na aktualizaci časovače
        }
\item \textit{Metoda: time\\_out}
\textit{Metoda na oznámení, že čas vypršel
        }
\item \textit{Metoda: check\\_answer}
\textit{Metoda na kontrolu odpovědi
        }
\item \textit{Metoda: kill\\_yourself}
\textit{Metoda pro ukončení okna galantní cestou
        }
\item \textit{Metoda: render\\_latex\\_to\\_katex}
\textit{Metoda na vytvoření rovnic ve formátu LaTeX

Args:
    latexEq (string): rovnice, které se mají vykreslit v LaTeXu}
\end{itemize}
\section*{Složka: resources}
\subsection*{Soubor: KatexHtmlTemplate.py}
\begin{itemize}
\end{itemize}
\section*{Složka: \_\_pycache\_\_}
\subsection*{Soubor: Tests.py}
\begin{itemize}
\end{itemize}
\section*{Složka: \_\_pycache\_\_}
\end{document}
